\documentclass[CJK,13pt]{beamer}
\input{macros.tex}
\def\courseurl{http://zhiqihuang.top/gr}

\def\tpage#1#2{
\title{GR \S{#1}  #2}
  \author{Zhiqi Huang}

\begin{frame}
\begin{center}
{\bf \Huge G}eneral {\bf \Huge R}elativity

{\vskip 0.1in}



{\Large \S #1 #2}

{\vskip 0.2in}

{Lecturer: 黄志琦}

\vskip 0.2in

\courseurl

\end{center}
\end{frame}
}


  \date{}
  \begin{document}
  \bch
\tpage{8}{Geodesic}


\begin{frame}
  
  \frametitle{这一讲的主题是:不会拐弯的憨憨在曲面上如何行走?}

  \addfig{1.5}{cliff.jpg}
  
\end{frame}



\begin{frame}
  \frametitle{思考题}

  \addfig{2}{geodesic.jpg}

  如图,假设有一根弹性杆两端固定在碗沿的相对两点上。杆很容易弯曲但很难拉长,怎么尽可能省力地压弹性杆使它贴住碗面?

\end{frame}


\begin{frame}
  
  当碗很深时,结果很不确定(很可能往边上压反而更容易)。但是当碗很浅时,你应该会毫不犹豫地往碗底压杆。

  \addfig{2}{cutgeodesic.jpg}

  抽象地说,如果在曲面上有足够相近的两点 $P,Q$ (距离 $\ll $ 附近的最小曲率半径), 则 $P, Q$ 间的{\blue 短程线的曲率向量和曲面的法向量重合}。也就是说:可以通过$P$, $Q$ 两点作曲面的法平面,法平面在曲面上切出来的曲线即是短程线。

  
\end{frame}


\begin{frame}
  \frametitle{测地线(geodesic)}
  把一段段这样的短程线光滑地拼接起来,就可以得到{\blue 测地线(geodesic)},它的曲率向量处处和曲面的法向量重合。{\scriptsize(请不要跟我杠直线这个特殊情况,让聊天轻松些……)}
  
  当然,由一段段局域的短程线拼成的测地线未必是全局的短程线。{\scriptsize(但反过来是成立的:全局的短程线必须是测地线。)}

  \addfig{1}{sphere_geodesic.jpg}

  例如在球面上,连接$A$, $B$两点的红色短弧,和从后面绕过来的灰色长弧都是测地线,显然红色那条才是短程线。
\end{frame}


\begin{frame}

  前面的讨论可能和街边大爷聊天差不多。

  \addfig{2}{dayechat.jpg}

  下面转入稍正经一点论述——
\end{frame}



\begin{frame}
  设曲面坐标为 $(u^1, u^2)$,测地线的弧长参数为 $s$。测地线的单位切矢量 $\vecT$ 的逆变形式为 $\frac{du^i}{ds}$ (请思考为什么)。

  
  当考察点沿着测地线移动时,由于测地线的密切平面一直保持和曲面垂直,测地线的单位切矢量从三维空间看一直沿着曲面的法向上下摆动。于是生活在曲面上的爬虫完全看不出测地线的单位切矢量 $\vecT$ 在变化。写成数学形式就是 $d\vecT = 0.$ 又根据逆变矢量的协变微分公式:

  $$0 = (d\vecT)^i = d \left(\frac{du^i}{ds}\right) + \Gamma^i_{\ jk}\frac{du^j}{ds} du^k = 0. $$
  两边同除以 $ds$,即得到{\blue 测地线方程:

  $$\frac{d^2u^i}{ds^2} + \Gamma^i_{\ jk}\frac{du^j}{ds}\frac{du^k}{ds} = 0.$$}
   
\end{frame}


\begin{frame}
  \frametitle{自由粒子运动方程即测地线}
  以后我们讨论四维时空 $(x^0, x^1, x^2, x^3)$ 时,{\blue 测地线方程
  $$\frac{d^2x^\mu}{ds^2} + \Gamma^\mu_{\ \nu\lambda}\frac{du^\nu}{ds}\frac{dx^\lambda}{ds} = 0.$$
    描述的是静止量非零的自由粒子的世界线。


    像光子这样没有静止质量的粒子的世界线不存在弧长参数,要用 $ds^2=0$,也就是 $g_{\mu\nu}dx^\mu dx^\nu=0$ 来描述}。
\end{frame}


\begin{frame}
  \frametitle{最后我们来证明测地线的长度取到稳定值,即变分为零}
  设曲线弧长参数为 $s$,则曲线的长度积分为
  $$ \int ds = \int g_{ij}\frac{du^i}{ds}\frac{du^j}{ds}  ds = \int L ds.$$
  这里的拉氏量
  $$L = g_{ij}\frac{du^i}{ds}\frac{du^j}{ds} .$$
  对 $u^k$ 运用 欧拉-拉格朗日方程
  $$\frac{d}{ds}\left(2g_{jk}\frac{du^j}{ds}\right) = g_{ij, k}\frac{du^i}{ds}\frac{du^j}{ds}.$$
  展开左边的微分,并两边乘以 $\frac{1}{2}g^{ik}$,稍加整理即可得到测地线方程(需用到联络的计算公式)。
\end{frame}


\begin{frame}

  这是我们学过的变分法吗?
  
  \addfig{1}{blackq.jpg}
  
  我不大清楚上面这段证明最早源自何处,总之它让我困惑了很多年……
\end{frame}


\ech
\end{document}
