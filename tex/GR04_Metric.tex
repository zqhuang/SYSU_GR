\documentclass[CJK,13pt]{beamer}
\input{macros.tex}
\def\courseurl{http://zhiqihuang.top/gr}

\def\tpage#1#2{
\title{GR \S{#1}  #2}
  \author{Zhiqi Huang}

\begin{frame}
\begin{center}
{\bf \Huge G}eneral {\bf \Huge R}elativity

{\vskip 0.1in}



{\Large \S #1 #2}

{\vskip 0.2in}

{Lecturer: 黄志琦}

\vskip 0.2in

\courseurl

\end{center}
\end{frame}
}


  \date{}
  \begin{document}
  \bch
\tpage{3}{Embedded Surface}


\begin{frame}
  \frametitle{正则曲面(regular surface)}
  设$n$维欧氏空间有曲面 $\vecx(u, v) = \left(x_1(u, v), x_2(u, v), \ldots, x_n(u, v)\right)$。

  \skipline
  
  本讲中我们将始终假设这些函数至少三次可导,并且矢量 $\frac{\partial\vecx}{\partial u}$ 和 $\frac{\partial\vecx}{\partial v}$ 始终线性无关(即两者均非零且方向不同)。这样的曲面称为正则曲面。

  \skiplines

  {\scriptsize 严格来讲,上面定义的是“正则曲面片”。一个正则曲面可以由多块正则曲面片光滑地拼接起来,但整体未必可用统一的满足正则曲面片要求的 $\vecx(u,v)$ 参数形式来表述。在大多数情形下,为了表述简便,我们忽略这些名词的细节差异。}
\end{frame}

\begin{frame}
  \frametitle{曲面的第一种基本形式(First Fundamental Form)}
  当从曲面上一个点 $\vecx(u,v)$ 移动到一个邻近的点 $\vecx(u+du, v+dv)$,则有微分表达式:
  $$ d\vecx = \frac{\partial\vecx}{\partial u} du + \frac{\partial\vecx}{\partial v} dv. $$
  由此得到曲面的第一种基本形式:
  $$d\vecx^2 = \left(\frac{\partial\vecx}{\partial u}\right)^2 du^2 +2\left(\frac{\partial\vecx}{\partial u}\cdot\frac{\partial\vecx}{\partial v}\right)du dv+ \left(\frac{\partial\vecx}{\partial v}\right)^2 dv^2.$$
  线性代数的常识告诉我们:这个表达式在变量替换 $u,v\rightarrow \tilde{u}, \tilde{v}$ 下是不变量。
\end{frame}






\begin{frame}
  \frametitle{曲面的第二种基本形式(Second Fundamental Form)}
  
\end{frame}


\ech
\end{document}
