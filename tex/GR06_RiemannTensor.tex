\documentclass[CJK,13pt]{beamer}
\input{macros.tex}
\def\courseurl{http://zhiqihuang.top/gr}

\def\tpage#1#2{
\title{GR \S{#1}  #2}
  \author{Zhiqi Huang}

\begin{frame}
\begin{center}
{\bf \Huge G}eneral {\bf \Huge R}elativity

{\vskip 0.1in}



{\Large \S #1 #2}

{\vskip 0.2in}

{Lecturer: 黄志琦}

\vskip 0.2in

\courseurl

\end{center}
\end{frame}
}


  \date{}
  \begin{document}
  \bch
\tpage{6}{Riemann Tensor}



\begin{frame}
  \frametitle{黎曼张量(Riemann Tensor)}
  上一讲我们利用升级的秘密武器的最后一击,干净利索地证明了高斯定理。把这个过程稍加推广,就可以得到:
 {\small
   \bea
   && \beta_{il}\beta_{jk} - \beta_{ik}\beta_{jl} \newl
  &=& \left(\vecn\cdot\vecx_{,i,l}\right)\left(\vecn\cdot\vecx_{,j,k}\right) - \left(\vecn\cdot\vecx_{,i,k}\right)\left(\vecn\cdot\vecx_{,j,l}\right) \newl
  &=& \vecx_{,i,l}\cdot\vecx_{,j,k} - \left(\vecx_{,m}\cdot\vecx_{,i,l}\right)\left(\vecx^{,m}\cdot\vecx_{,j,k}\right) - \vecx_{,i,k}\cdot\vecx_{,j,l} + \left(\vecx_{,m}\cdot\vecx_{,i,k}\right)\left(\vecx^{,m}\cdot\vecx_{,j,l}\right)  \newl
  &=& \vecx_{,i,l}\cdot\vecx_{,j,k}  - \Gamma_{mil}\Gamma^m_{\ jk} - \vecx_{,i,k}\cdot\vecx_{,j,l}  + \Gamma_{mik}\Gamma^m_{\ jl}  \newl
  &=& \left(\vecx_{,i}\cdot\vecx_{,j, k}\right)_{,l}-  \left(\vecx_{,i}\cdot\vecx_{,j,l}\right)_{,k}   + \Gamma_{mik}\Gamma^m_{\ jl}   - \Gamma_{mil}\Gamma^m_{\ jk} \newl
  &=& \left(\Gamma_{ijk}\right)_{,l} -  \left(\Gamma_{ijl}\right)_{,k} + \Gamma_{mik}\Gamma^m_{\ jl} - \Gamma_{mil}\Gamma^m_{\ jk} \newl    
 \eea }
 我们把这个结果称为黎曼张量 (Riemann Tensor)
 \tbox{$$R_{ijkl} \equiv \left(\Gamma_{ijk}\right)_{,l} -  \left(\Gamma_{ijl}\right)_{,k} + \Gamma_{mik}\Gamma^m_{\ jl} - \Gamma_{mil}\Gamma^m_{\ jk}$$}
\end{frame}


\begin{frame}
  \frametitle{黎曼张量的对称性}
  由 $R_{ijkl}  = \beta_{il}\beta_{jk} - \beta_{ik}\beta_{jl}$ 容易得到:
  \tbox{$$ R_{ijkl} = R_{klij} = - R_{jikl} = - R_{ijlk}.$$
    $$ R_{ijkl} +  R_{iklj} +  R_{iljk} = 0.$$}  
  由此可以判断 $i=j$ 或者 $k=l$ 时黎曼张量为零。对二维曲面而言,独立的非零的黎曼张量只有 $R_{1221}=\det(\beta)$。我们似乎并没有得到新的等式。但是,我们今后会把度规、联络、黎曼张量推广到高维的空间,上述对称性仍然成立。
\end{frame}


\thinkf{请思考,在 $n$ 维空间,黎曼张量共有多少个独立的分量?}

\begin{frame}
  \frametitle{已经多次提到了“张量”的概念,是时候给出一个更加严谨的定义了}  
  当建立坐标系之后,可以把物理量投影到不同的基上,这些投影值叫做物理量的分量。{\blue 物理量的分量的值依赖于人为选取的坐标系},可以用带一系列带指标的符号来表示。

  \skipline

  \tbox{我们定义一个量 $T$ 为张量,如果它在任意坐标系 $(u^1, u^2,\ldots)$ 和 $(\tilde{u}^1, \tilde{u}^2,\ldots)$ 里的分量总是满足下列转换关系:
  $$ \tilde{T}_{i_1i_2\ldots}^{\ \  j_1j_2\ldots}= \frac{\partial u^{k_1}}{\partial \tilde{u}^{i_1}} \frac{\partial u^{k_2}}{\partial \tilde{u}^{i_2}}\ldots \frac{\partial \tilde{u}^{j_1}}{\partial u^{l_1}} \frac{\partial \tilde{u}^{j_2}}{\partial u^{l_2}} \ldots T_{k_1k_2\ldots}^{\ \ l_1l_2\ldots}.$$}

\end{frame}

\begin{frame}
  \frametitle{零阶、一阶和二阶的张量}
  我们比较熟悉的有带零个指标的“标量”(零阶张量)和带一个指标的“矢量”(1阶张量),像度规 $g_{ij}$ 这样的带两个指标的就叫二阶张量。

  我们来说明度规 $g_{ij}$ 为什么是张量。由于要求
  $$ ds^2 =  g_{ij}du^idu^j$$ 是不变量,当我们做一个坐标系的变换 $(u^1,u^2,\ldots)\rightarrow (\tilde{u}^1, \tilde{u}^2, \ldots)$
  时,在新坐标系里的度规 $\tilde{g}_{ij}$ 必须满足:
  $$ \tilde{g}_{ij}d\tilde{u}^id\tilde{u}^j =  g_{ij}du^idu^j.$$
  这是我们熟悉的线性代数的基的变换,很容易看出:
  $$ \tilde{g}_{ij} = \frac{\partial u^k}{\partial \tilde{u}^i} \frac{\partial u^l}{\partial \tilde{u}^j} g_{kl}.$$
  即度规满足张量的定义。
\end{frame}


\begin{frame}
  \frametitle{联络不是张量,黎曼张量是张量}
  请自行用定义证明联络 $\Gamma_{ijk}$ 不是张量,黎曼张量 $R_{ijkl}$ 是张量。

  \addfig{1.2}{caigou.jpg}
\end{frame}


\begin{frame}
  \frametitle{张量的指标升降}

  对张量我们可以沿用之前定义的指标升降规则。

  \addfig{1}{eg.jpg}

  把黎曼张量 $R_{ijkl}$ 第一个指标升上来 $R^i_{\ jkl} \equiv g^{im}R_{mjkl}$.
\end{frame}



\begin{frame}
  \frametitle{从已有张量构造新张量的第1种办法:张量积}
  把一个 $m$ 阶张量和一个 $n$ 阶张量 的分量两两相乘,就可以得到 $m+n$ 阶张量。

  \addfig{1}{eg.jpg}

  矢量 $A^i$ 和度规 $g_{jk}$ 的张量积 是三阶张量 $A^ig_{jk}$
\end{frame}


\begin{frame}
  \frametitle{从已有张量构造新张量的第2种办法:收缩}
  把一个 $m$ 阶张量的一组上下指标取成相同,按爱因斯坦求和规则求和之后,可以得到一个 $m-2$ 阶张量。

  \addfig{1}{eg.jpg}

  对黎曼张量 $R^i_{jkl}$ 的第一个指标 $i$ 和最后一个指标 $l$ 进行收缩,得到{\blue 里奇(Ricci)张量: $R_{jk} \equiv R^i_{\ jki}$};把里奇张量的两个指标收缩,得到{\blue 里奇标量(Ricci scalar)  $ R = R^j_{\ j}.$}
\end{frame}

\thinkc{三维欧氏空间里的曲面的里奇标量 $R$ 和高斯曲率 $K$ 的关系是什么?}

\begin{frame}
  \frametitle{从已有张量构造新张量的第3种办法:求协变导数}
  一般来说,对大于零阶的张量求坐标偏导,会使张量变成非张量。在张量分析中,有新的一种导数叫做协变导数。对任意阶张量求协变导数后会得到高一阶的张量。我们会在下一讲中讲解协变导数的几何意义和计算规则。

  \addfig{1}{neg.jpg}

\end{frame}


\ech
\end{document}
