\documentclass[CJK,13pt]{beamer}
\input{macros.tex}
\def\courseurl{http://zhiqihuang.top/gr}

\def\tpage#1#2{
\title{GR \S{#1}  #2}
  \author{Zhiqi Huang}

\begin{frame}
\begin{center}
{\bf \Huge G}eneral {\bf \Huge R}elativity

{\vskip 0.1in}



{\Large \S #1 #2}

{\vskip 0.2in}

{Lecturer: 黄志琦}

\vskip 0.2in

\courseurl

\end{center}
\end{frame}
}


  \date{}
  \begin{document}
  \bch
  \tpage{19}{General Discussion about Black Holes}

  \begin{frame}
    我无意在幼儿版GR中把大家拉黑(拉入黑洞知识的泥潭),这一讲的唯一目的是让你不至于在听有关黑洞的讲座时全程梦游。

    \addfig{1.2}{mengyou.jpg}
  \end{frame}


  \secpage{Kerr Black Hole}{    \bea
    ds^2 &=& \left(1-\frac{2GMr}{r^2+a^2\cos^2\theta}\right) dt^2 \newl    
    && + \frac{4GMra\sin^2\theta}{r^2+a^2\cos^2\theta} dt d\phi \newl
    && -\left(r^2+a^2+\frac{2GMra^2\sin^2\theta}{r^2+a^2\cos^2\theta}\right)\sin^2\theta d\phi^2 \newl
    && - \frac{r^2+a^2\cos^2\theta}{r^2-2GMr + a^2}dr^2 \newl
    && - \left(r^2+a^2\cos^2\theta\right)d\theta^2
    \eea
}
  
  \begin{frame}
    \frametitle{Kerr Black Hole}
    黑洞附近的气体和其他天体会被黑洞吞噬,吞噬的过程中黑洞会吸收它们的轨道角动量。所以宇宙中的黑洞应该都或多或少有些“自转”。这种黑洞叫做 Kerr 黑洞。


    \addfig{1}{RealBH.jpg}
  \end{frame}

  \begin{frame}
    如果取其“自转轴”为南北极方向建立“球坐标系”,质量为 $M$,单位质量的角动量为 $a$ 的Kerr黑洞的度规为
{\blue
    \bea
    ds^2 &=& \left(1-\frac{2GMr}{r^2+a^2\cos^2\theta}\right) dt^2 \newl    
    && + \frac{4GMra\sin^2\theta}{r^2+a^2\cos^2\theta} dt d\phi \newl
    && -\left(r^2+a^2+\frac{2GMra^2\sin^2\theta}{r^2+a^2\cos^2\theta}\right)\sin^2\theta d\phi^2 \newl
    && - \frac{r^2+a^2\cos^2\theta}{r^2-2GMr + a^2}dr^2 \newl
    && - \left(r^2+a^2\cos^2\theta\right)d\theta^2
    \eea
}
(哇!五行就写完了!)
  \end{frame}


  \begin{frame}
    \frametitle{因为Kerr度规相关的表达式都太长了,我们要约定一些符号来尽可能地保护视力。}
    \bitem
  \item{定义符号 {\blue $ \rho^2 \equiv r^2+a^2\cos^2\theta $}}
  \item{定义符号 {\blue $ \Delta \equiv r^2-2rGM+a^2 $} ($\Delta$ 是长度平方量纲)}
    \eitem

  \end{frame}


  \begin{frame}
    
    这样Kerr黑洞的度规可以按 $(t, r, \theta,\phi)$ 的坐标次序写为:
    \be
    g_{\mu\nu} = \left[\begin{matrix}    
        1-\frac{2GMr}{\rho^2} &  &  & \frac{2GMra\sin^2\theta}{\rho^2} \\
         & - \frac{\rho^2}{\Delta} &  &  \\
         &  & - \rho^2 &  \\
        \frac{2GMra\sin^2\theta}{\rho^2} &   &  & -\left(r^2+a^2+\frac{2GMra^2\sin^2\theta}{\rho^2}\right)\sin^2\theta 
      \end{matrix}\right]
    \ee
    
  \end{frame}

  \begin{frame}
    \frametitle{逆变形式的度规}

    \be
    g^{\mu\nu} = \left[\begin{matrix}
        \frac{2 GM a^2 r \sin^2\theta   + (r^2+a^2)\rho^2}{  \rho^2\Delta} &  &  & \frac{2 GM a r}{\rho^2\Delta}\\
         & -\frac{\Delta}{\rho^2} &  & \\
         &  & - \frac{1}{\rho^2} & \\
        \frac{2 GM a r}{\rho^2\Delta} &  &  & \frac{2 GM r - \rho^2}{\rho^2\Delta \sin^2\theta }
      \end{matrix}\right]
    \ee

  \end{frame}
  

  \begin{frame}
    \frametitle{Kerr几何}
    \addfig{3.8}{KerrGeometry.png}
  \end{frame}  


  \begin{frame}
    \frametitle{Kerr度规里的测试粒子}
    度规 $g_{\mu\nu}$ 仍然不依赖于 $t$ 和 $\phi$,所以 $p_t$ 和 $p_\phi$ 守恒。

    \skipline
    
    此外,$ds^2$ 表达式照例会给我们一个方程。

    \skipline
    
    由于失去了空间对称性,{\blue 不能再假设测试粒子在一个平面上运动}(除非刚好粒子位置和初速度都在赤道平面 $\theta=\frac{\pi}{2}$ 上)。

    \skipline
    
    还缺的一个方程只能用最原始的测地线方程来凑。

    \skipline
    
    为此我们要计算联络……


  \end{frame}
  

  \begin{frame}
    \addfig{3.8}{Kerrconn.png}
  \end{frame}


  \begin{frame}
    虽然完全算不动,但是为了能吹几句——
  \end{frame}


  \begin{frame}
    \frametitle{引力拖曳效应}
    在赤道( $\theta=\frac{\pi}{2}$) 平面上运动的粒子,$p_\phi$ 守恒可以写成:
    $$ -\frac{2GMa}{r}\frac{dt}{ds}+\left(r^2+a^2+\frac{2GMa^2}{r}\right)\frac{d\phi}{ds} = \frac{L}{m}$$    
    考虑一个非常简化的情形,$r\gg GM \gg a $,且粒子运动速度远小于光速。也就是说,Kerr黑洞的角动量比较小,且测试粒子离得比较远,运动是非相对论的。这时取最低阶近似,有
    $$ \frac{d\phi}{dt} \approx \frac{\frac{L}{m}+\frac{2GMa}{r}}{r^2+a^2}$$
    可以看到,即使粒子的守恒角动量为零(比如它从很远处瞄准黑洞中心下落过来),它也会被黑洞的自旋带着旋转起来,旋转方向和黑洞自旋方向一致。这就是引力拖曳效应。
  \end{frame}
  
  \begin{frame}
    \frametitle{$a>GM$的Kerr度规存在吗?}
    物理学家们猜想有一条“宇宙监督”(cosmic censorship) 法则,使得所有的奇点都被保护在视界之内。

    \skipline
    
    当 $a>GM$ Kerr度规的奇点就会失去视界的“保护”。这种时空“很难”(因为不好严格证明,所以只能这么说)形成,因为当单位质量角动量过大时,物质不容易掉进黑洞。所以至少定性地说,“宇宙监督法则”是有些根据的猜测。
  \end{frame}  


    \begin{frame}
      \frametitle{带电的黑洞}
      稳态的黑洞除了质量,角动量,原则上来讲还可以有电荷$Q$ (即所谓的稳态黑洞只有三个参数的“无毛定理”)。

      这时只要把Kerr度规的
      $$\Delta \equiv r^2+a^2-2GMr$$
      换为
      $$\Delta \equiv r^2+a^2-2GMr+GQ^2$$
      这样得到的度规称为 Kerr-Newman 度规。它是最一般的稳定黑洞。


      \skipline

      相关知识可以参考

      \url{https://arxiv.org/pdf/1410.6626.pdf}
  \end{frame}  

    \secpage{黑洞辐射}{$$ T = \frac{\hbar c^3}{8\pi k_BGM} $$}
    
    \begin{frame}
      我们来估算下质量为 $M$ 的史瓦西黑洞的“视界温度”:黑洞把粒子禁锢在 $\Delta t\sim \frac{4\pi GM}{c^3}$ 大小的时间区域内(注意 $r<2GM$ 时,$r$ 是类时坐标),根据海森堡不确定原理,黑洞视界内的粒子满足
      $$\Delta E \Delta t \gtrsim \frac{\hbar}{2}$$
      于是可以估算出粒子能量的量子波动
      $$\Delta E \sim \frac{\hbar c^3}{8\pi GM}$$
      那么史瓦西黑洞的``温度''
      $$T \sim \frac{\Delta E}{k_B} \sim \frac{\hbar c^3}{8\pi k_BGM}$$      
      ……好了这种瞎扯实在太不严肃,我编不下去了。
    \end{frame}

    \begin{frame}
      \frametitle{正经的结论}
      史瓦西黑洞的“霍金温度”
      {\blue $$ T = \frac{\hbar c^3}{8\pi k_BGM} $$}
      严肃的证明要在很厚的书里才能找到,我们幼儿版GR直接略去……

      \skipline
      
      简单估算一下,太阳质量黑洞的霍金温度大约为 $4\times 10^{-7}\mathrm{K}$,好像完全可以忽略。

      但是,小质量黑洞的霍金温度就会更高,由此产生的辐射会使小质量黑洞逐渐损失质量,这就是“黑洞蒸发”效应。
    \end{frame}

    \begin{frame}
      \frametitle{史瓦西黑洞的辐射功率}
      回忆一下黑体单位表面积的辐射功率为:
      $$\frac{dP}{dS} = \frac{\pi^2k_B^4}{60\hbar^3c^2} T^4$$
      贯彻我们的瞎扯精神,把黑洞当成黑体(反正就差一个字\lfig{0.2}{emoji_wulian.jpg}),把黑洞视界当成“表面”,史瓦西黑洞的辐射功率为:
      $$ P =  4\pi \left(\frac{2GM}{c^2}\right)^2 \frac{\pi^2k_B^4}{60\hbar^3c^2}  \left(\frac{\hbar c^3}{8\pi k_BGM}\right)^4 =\frac{c^6\hbar}{15360\pi G^2M^2}$$
      虽然推导是瞎编的,但是结论仍然正确:
      {\blue $$P = \frac{c^6\hbar}{15360\pi G^2M^2}$$}
    \end{frame}

    \begin{frame}
      \frametitle{史瓦西黑洞的蒸发}
      我们继续按照“能量守恒”编出下列史瓦西黑洞的蒸发公式:
      $$\frac{d\left(Mc^2\right)}{dt} = -P = - \frac{c^6\hbar}{15360\pi G^2M^2}$$
      这里的 $t$ 是史瓦西坐标系里的 $t$,也就是无穷远处相对于坐标系静止的观测者的固有时间。

      把蒸发公式等价地写成:
      $$\frac{d(M^3)}{dt} = -P = - \frac{c^4\hbar}{5120\pi G^2}$$
      可见质量为 $M$ 的史瓦西黑洞的寿命为
     {\blue $$t = \frac{5120\pi G^2M^3}{c^4\hbar}$$}      
    \end{frame}


    \begin{frame}
      \frametitle{原初黑洞的质量下限}
      把黑洞寿命等价地写成:
      $$\frac{t}{\mathrm{Gyr}} = 2.67 \left(\frac{M}{10^{11}\mathrm{kg}}\right)^3 $$
      这里的 $\mathrm{Gyr}$ 是 $10^9$ 年。

      \skiplines

      如果按照标准宇宙学模型的观点,宇宙的年龄大概是 $13.7\mathrm{Gyr}$。那么我们能看到的来自宇宙早期的史瓦西黑洞就有个$\sim 10^{11}\mathrm{kg}$ 的质量下限。
    \end{frame}    
    

    \ech
\end{document}




  
