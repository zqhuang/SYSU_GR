\documentclass[CJK,13pt]{beamer}
\input{macros.tex}
\def\courseurl{http://zhiqihuang.top/gr}

\def\tpage#1#2{
\title{GR \S{#1}  #2}
  \author{Zhiqi Huang}

\begin{frame}
\begin{center}
{\bf \Huge G}eneral {\bf \Huge R}elativity

{\vskip 0.1in}



{\Large \S #1 #2}

{\vskip 0.2in}

{Lecturer: 黄志琦}

\vskip 0.2in

\courseurl

\end{center}
\end{frame}
}


  \date{}
  \begin{document}
  \bch
  \tpage{12}{Energy Momentum Tensor}


  \begin{frame}
    \frametitle{{\blue 物质} 的 {\blue 能量} {\blue 动量} {\blue 张量}}

    \addfig{1}{blackq.jpg}
    
    这四个词我都知道是什么意思,但合在一起……
  \end{frame}

  \secpage{关于符号的一点点疑惑}{学广义相对论最难搞清楚的不是指标,而是正负号}

  \begin{frame}
    上一讲我们给出了能量动量张量的定义:
    \tbox{$$T_{\mu\nu} = 2\frac{\delta \mathcal{L}_m}{\delta g^{\mu\nu}} -g_{\mu\nu}\mathcal{L}_m$$}
    或者等价地
    \tbox{$$T_{\mu\nu} = \frac{2}{\sqrt{-g}}\frac{\delta \left(\sqrt{-g}\mathcal{L}_m\right)}{\delta g^{\mu\nu}} $$}    
  \end{frame}


  \begin{frame}
    我们也可以以逆变的形式定义:
    \tbox{$$T^{\mu\nu} = -2\frac{\delta \mathcal{L}_m}{\delta g_{\mu\nu}} -g^{\mu\nu}\mathcal{L}_m$$}
    或者等价地
    \tbox{$$T^{\mu\nu} = -\frac{2}{\sqrt{-g}}\frac{\delta \left(\sqrt{-g}\mathcal{L}_m\right)}{\delta g_{\mu\nu}} $$}        
    这里的各种负号让你困惑了吗?
  \end{frame}
    
  \begin{frame}
    我们先来看一个恒等式:
    $$ g^{\mu\rho}g_{\rho\lambda} = g^{\mu}_{\ \lambda}$$
    两边取变分,右边是个常数,变分为零,所以
    $$ \delta g^{\mu\rho}g_{\rho\lambda}+g^{\mu\rho}\delta g_{\rho\lambda} = 0 $$
    两边同乘以 $g^{\nu\lambda}$,并把第二项移到方程右边,得到
    \tbox{$$ \delta g^{\mu\nu} = - g^{\mu\rho}g^{\nu\lambda}\delta g_{\rho\lambda}$$}
    类似地可以得到
    \tbox{$$ \delta g_{\mu\nu} = - g_{\mu\rho}g_{\nu\lambda}\delta g^{\rho\lambda}$$}    
  \end{frame}


  \begin{frame}
    也就是说:
    
    {\blue $\delta g_{\mu\nu}$ 和 $\delta g^{\mu\nu}$ 之间可以进行指标升降,但要多个负号。}

    于是有

    \bea
    \delta \mathcal{L}_m &=& f_{\mu\nu}\delta g^{\mu\nu} \newl
    &=& f^{\alpha\beta}g_{\alpha\mu}g_{\beta\nu}\delta g^{\mu\nu} \newl
    &=& -f^{\alpha\beta}\delta g_{\alpha\beta}
    \eea
    这里的 $f_{\mu\nu} = \frac{1}{2}\left(T_{\mu\nu}+g_{\mu\nu}\mathcal{L}_m\right)$。

    然后就有
    \bea
    T^{\alpha\beta} &=& 2 f^{\alpha\beta}  - g^{\alpha\beta}\mathcal{L}_m \newl
    &=& -2\frac{\delta\mathcal{L}_m}{\delta g_{\alpha\beta}} - g^{\alpha\beta}\mathcal{L}_m 
    \eea
    
  \end{frame}

  
  \begin{frame}
    \bcenter
    \lfig{1}{chulecai.jpg}
    
    这种数学推导除了让本喵更迷惑之外毫无用处……
    \ecenter
  \end{frame}


  \secpage{一个奇怪的标量}{长这么丑竟然也是……}

  \begin{frame}
    我们把四维坐标 $x^\mu$ 分离为 $t=x^0$ 和 $\vecx=(x^1,x^2,x^3)$ (注意这里的 $\vecx$ 只具有数学意义,不代表客观物理矢量),试说明
    设某静止质量为 $m$ 的粒子的世界线方程为 $\vecx = \vecy(t)$,四维动量为 $p^\mu=m\frac{dx^\mu}{ds}$ ($ds=\sqrt{g_{\mu\nu}dx^\mu dx^\nu}$是弧长参数)。试说明    
    $$\frac{\delta^{(3)}(\vecx-\vecy)}{p^0\sqrt{-g}}$$
    是标量,这里的 $\delta^{(3)}(\vecx-\vecy)$ 是(数学意义上的)三维Dirac函数。

    \addfig{1.5}{choubiaoliang.jpg}
  \end{frame}
  
  \secpage{一个粒子的$T_{\mu\nu}$}{$$ T^{\mu\nu} =  p^\mu p^\nu \frac{\delta^{(3)}\left(\vecx - \vecy\right)}{p^0\sqrt{-g}}$$}
  
  \begin{frame}
    \frametitle{粒子的 $T^{\mu\nu}$}    
    一个质量为 $m$ 的自由粒子,我们不妨称之为Bob,它有作用量
    $$ S_m = -m \int ds $$
    积分沿粒子的世界线进行,$s$ 为弧长参数。

    \skipline

    假设 Bob 的世界线用时间坐标 $t=x^0$ 描述为 $y^\mu(t)$,其四维动量为 $p^\mu = m\frac{dx^\mu}{dt}$,
    $$ S_m = -m \int \sqrt{g_{\mu\nu}\left(y(t)\right) \frac{dy^\mu}{dt}\frac{dy^\nu}{dt}}dt$$
    把粒子的四维动量记作 $p^\mu$,上式可以写成
    $$ S_m = - m\int \sqrt{g_{\mu\nu}\left(y(t)\right) p^\mu p^\nu}\frac{dt}{p^0}$$
  \end{frame}


  \begin{frame}
    \frametitle{粒子的 $T^{\mu\nu}$(续)}    
    利用三维 $\delta$ 函数,Bob的作用量可以写成:
    $$ S_m = - m\int \sqrt{g_{\mu\nu}\left(x\right)p^\mu p^\nu} \frac{\delta^{(3)}\left(\vecx - \vecy\right)}{p^0}d^3\vecx dt$$
    把这个式子河蟹一下:
    $$ S_m = - m\int \sqrt{g_{\mu\nu}\left(x\right)p^\mu p^\nu} \frac{\delta^{(3)}\left(\vecx - \vecy\right)}{p^0\sqrt{-g}} \sqrt{-g} d^4x$$
    于是 $\mathcal{L}_m = -  m\sqrt{g_{\mu\nu}\left(x\right)p^\mu p^\nu} \frac{\delta^{(3)}\left(\vecx - \vecy\right)}{p^0\sqrt{-g}}$.
    
    {\scriptsize 要说明一下,这里四维坐标系是给定的,作用量里采用的广义坐标是对应于每一个 $t$ 的 $y^\mu(t)$,以及对应于每一个四维点 $x$ 的 $g_{\mu\nu}(x)$。我们曾经固定$g_{\mu\nu}$,变动 $y^\mu(t)$,推导了粒子的运动方程;现在我们固定 $y^\mu(t)$,变动 $g_{\mu\nu}$,来推导爱因斯坦方程里的 $T^{\mu\nu}$(如果你要推导完整的爱因斯坦方程,当然需要把依赖于 $g_{\mu\nu}$ 的时空作用量加进来一起做变分)。}    
  \end{frame}

  \begin{frame}
    \frametitle{粒子的 $T^{\mu\nu}$(续)}
    最后,

    \bea
    T^{\mu\nu} &=& -\frac{2}{\sqrt{-g}}\frac{\delta\left(\sqrt{-g}\mathcal{L}_m\right)}{\delta g_{\mu\nu}} \newl
    &=& m\frac{p^\mu p^\nu}{ \sqrt{g_{\mu\nu}\left(x\right)p^\mu p^\nu}} \frac{\delta^{(3)}\left(\vecx - \vecy\right)}{p^0\sqrt{-g}}
    \eea
    由于已经对 $g_{\mu\nu}$ 取完了变分,我们可以取 $g_{\mu\nu}$ 为“该取的值”了,也就是有 $g_{\mu\nu}p^\mu p^\nu = p^\mu p_\mu = m^2$ ,{\blue 最后我们得到单粒子的能量动量张量:
   $$ T^{\mu\nu} =  p^\mu p^\nu \frac{\delta^{(3)}\left(\vecx - \vecy\right)}{p^0\sqrt{-g}}$$}
  \end{frame}


  \begin{frame}
    \frametitle{粒子的 $T^{\mu\nu}$(续)}
    如果某四维点 $x$ 附近,有一堆粒子作{\blue 统计上各向同性的随机运动},则
    $$T^{0i} \propto \sum p^i = 0$$
    且
    $$T^{ij} \propto \sum p^i p^j \propto \delta^{ij}$$
    即能量动量张量是对角矩阵,且 $T^{11}=T^{22}=T^{33}$。


    \skipline
    
    和平直时空的非相对论理想气体比较,我们可以把 $T^{00}$ 叫做能量密度,把 $T^{11}=T^{22}=T^{33}$ 叫做压强。

  \end{frame}
  

  \secpage{理想流体的 $T^{\mu\nu}$}{$$T^{\mu\nu} = (p+\rho)u^\mu u^\nu - p g^{\mu\nu}$$}

  \begin{frame}
    \frametitle{理想流体的 $T^{\mu\nu}$}
    “理想流体”的每一个四维时空点 $x$ 都有一个“宏观速度” $u^\mu(x)$,和流体宏观速度共动的观测者看到附近的流体是各向同性的。流体的实质当然是一堆有微观随机运动的粒子,所以流体的共动参照系里流体的能量动量张量为:
    \be
    \left.\tilde{T}^{\mu\nu}\right\vert_{\rm comoving} = \begin{pmatrix}
      \rho & 0 & 0 & 0 \\
      0 & p & 0 & 0 \\
      0 & 0 & p & 0 \\
      0 & 0 & 0 & p      
    \end{pmatrix}
    \ee
    这里的 $\rho$ 是能量密度,$p$ 是压强(均指共动参考系里观测到的)。
  \end{frame}

  
  
  \begin{frame}
    \frametitle{理想流体的 $T^{\mu\nu}$}    
    转换到原先的参照系,我们先给结论: 理想流体的能量动量张量是
     \tbox{$$T^{\mu\nu} =  (\rho+p)u^\mu u^\nu - p g^{\mu\nu}$$}
    {\scriptsize 如果你采用 $-+++$ 度规符号习惯,上面的右边第二项的减号要变成加号。}

    \skiplines

     如何证明这个 $T^{\mu\nu}$ 在共动参考系里具有 $\diag(\rho, p, p, p)$ 的形式呢?下一讲我们将详细讨论观测量的问题。
  \end{frame}


  
  
  
\ech
\end{document}




  
