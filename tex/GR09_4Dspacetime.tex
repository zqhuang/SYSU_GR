\documentclass[CJK,13pt]{beamer}
\input{macros.tex}
\def\courseurl{http://zhiqihuang.top/gr}

\def\tpage#1#2{
\title{GR \S{#1}  #2}
  \author{Zhiqi Huang}

\begin{frame}
\begin{center}
{\bf \Huge G}eneral {\bf \Huge R}elativity

{\vskip 0.1in}



{\Large \S #1 #2}

{\vskip 0.2in}

{Lecturer: 黄志琦}

\vskip 0.2in

\courseurl

\end{center}
\end{frame}
}


  \date{}
  \begin{document}
  \bch
\tpage{9}{Four-Dimensional Spacetime}


\begin{frame}
  \frametitle{从二维曲面到四维时空}

  二维曲面的内蕴几何学已经具有足够的代表性。在大多数情况下阿,可以直接把二维曲面的内蕴几何学的结论推广到四维时空。但是,我们之前很多的推导都借助了嵌入高维平直空间的假设,下面我们再以尽量不借助高维平直空间的形式过一遍这些概念,以加深对广义相对论的逻辑架构的理解。

  \skiplines

  为了和大多数文献保持一致,我们将使用 $(x^0, x^1, x^2, x^3)$ 来标记四维时空的坐标。这里的 $x^0$ 经常具有类似于“时间”的意义——但要注意弯曲时空的时间和空间坐标选取具有更大的随意性,一般不能直接用狭义相对论那套“默认上帝视角”的术语来描述问题。

  {\scriptsize 你或者清晰地说“xx事件在xx坐标系的时间坐标”,或者明确地描述事件“xx通过望远镜观测到xx钟的时间”,模棱两可的描述“对xx观测者来说xx的钟走得更慢”在广义相对论里是不被认可的。}

\end{frame}

\secpage{度规}{$$ds^2=g_{\mu\nu}dx^\mu dx^\nu$$}

\begin{frame}
  \frametitle{Minkowski时空}
  狭义相对论里的 Minkowski 时空是四维时空的一种特殊情况。如果选取惯性运动者的固连直角坐标系,Minkowski 时空的第一基本形式——以后我们简单就说它的度规是:
  \be
  ds^2 = (dx^0)^2 - (dx^1)^2 - (dx^2)^2 - (dx^3)^2.
  \ee
  也就是度规矩阵是个对角矩阵 $\diag(1,-1,-1,-1)$。

  \skipline
  
  {\scriptsize 在很多文献中,也有把 Minkowski 时空的度规取为 $\diag(-1, 1, 1, 1)$ 的。两种取法并无谁对谁错之分,但是在同一个工作里,你最好始终坚持用一种取法,否则计算中很容易产生符号错误。}
  
\end{frame}


\begin{frame}
  \frametitle{Minkowski度规}
  {\blue 由对角矩阵 $\diag(1,-1,-1,-1)$ 定义的 Minkowski 度规写作 $\eta_{\mu\nu}$。}
  这样在 Minkowski 时空 $ ds^2 = \eta_{\mu\nu} dx^\mu dx^\nu.$
  
  \addfig{1.9}{yueyueyue.jpg}
  
  从这一讲开始,我们将约定:{\blue 默认用希腊字母来标记四维时空指标,用拉丁字母来标记三维空间指标。} 例如 $p^\mu p_\mu$ 默认表示 $p^0p_0+p^1p_1+p^2p_2+p^3p_3$,而 $p^ip_i$ 默认表示 $p^1p_1 + p^2p_2 + p^3p_3$。
\end{frame}


\begin{frame}
  \frametitle{一般四维时空度规}
  \bmini{0.5}
  一般的四维时空是可以有内禀弯曲的,也就是说,无论如何取坐标系,度规 $g_{\mu\nu}$ 都不能处处和 $\eta_{\mu\nu}$ 相同:
  $$ ds^2= g_{\mu\nu}dx^\mu dx^\nu.$$
  \emini
  \bmini{0.45}
  \addfig{1.5}{curvedspace.jpg}
  \emini

  \skipline
  
  在宇宙的大多数地方,时空都比较接近 Minkowski 时空,可以把 $g_{\mu\nu}$ 写成
  \be
  g_{\mu\nu} = \eta_{\mu\nu} + h_{\mu\nu}.
  \ee
  这里的 $\lvert h_{\mu\nu}\rvert\ll 1$. 我们后面会专门对这种形式的度规进行研究,并发展一些非常有用的计算方法。

  \skipline

  在另外一些极端的地方,如黑洞、中子星等致密星体的周围。时空和 Minkowski 时空差距巨大。这种问题一般很难求解,我们将仅讨论一些最简单的情况。
  
\end{frame}


\secpage{联络}{$$\Gamma_{\lambda\mu\nu} = \frac{1}{2}\left(g_{\lambda\mu,\nu}+g_{\lambda\nu,\mu}-g_{\mu\nu,\lambda}\right).$$}

\begin{frame}
  我们先不假设度规的存在(因而也没有指标的升降等操作)。协变和逆变张量纯粹由数学定义给出,即它们满足张量的坐标变换性质
  {\blue  $$ \widetilde{T}_{\mu_1\mu_2\ldots}^{\ \  \nu_1\nu_2\ldots}= \frac{\partial x^{\lambda_1}}{\partial \tilde{x}^{\mu_1}} \frac{\partial x^{\lambda_2}}{\partial \tilde{x}^{\mu_2}}\ldots \frac{\partial \tilde{x}^{\nu_1}}{\partial x^{\rho_1}} \frac{\partial \tilde{x}^{\nu_2}}{\partial x^{\rho_2}} \ldots T_{\lambda_1\lambda_2\ldots}^{\ \ \rho_1\rho_2\ldots}.$$}
    即可。

    \skipline
    
    在此前提下,我们根本不讨论物理实体,而是把协变张量和逆变张量看成不同的数学对象。
\end{frame}


\begin{frame}
  \frametitle{联络的坐标变换性质}
  如果要求协变微商保持张量性,就可以推出,当从一个坐标系 $(x^0,x^1,x^2,x^3)$ 变换到另一个坐标系 $(\tilde{x}^0,\tilde{x}^1,\tilde{x}^2,\tilde{x}^3)$,联络的变换规则为:
  {\blue $$\tilde{\Gamma}^\lambda_{\ \mu\nu} = \frac{\partial \tilde{x}^\lambda}{\partial x^\alpha}\frac{\partial^2x^\alpha}{\partial \tilde{x}^\mu\partial \tilde{x}^\nu} + \Gamma^\alpha_{\ \beta\gamma}\frac{\partial \tilde{x}^\lambda}{\partial x^\alpha}\frac{\partial x^\beta}{\partial \tilde{x}^\mu}\frac{\partial x^\gamma}{\partial \tilde{x}^\nu} $$}
  由此可见,联络不是一个张量。

  \skipline

  如果有多种可能的联络,显然{\blue 任何两个联络的差是张量} (因为多余的那项 $\frac{\partial \tilde{x}^\lambda}{\partial x^\alpha}\frac{\partial^2x^\alpha}{\partial \tilde{x}^\mu\partial \tilde{x}^\nu}$ 抵消了)。反过来,{\blue 给联络 $\Gamma^\lambda_{\ \mu\nu}$ 加上任何张量 $B^\lambda_{\ \mu\nu}$ 都不会影响联络的坐标变换规则。}
\end{frame}



\begin{frame}
  \frametitle{协变微商的保张量性}
  我们再来回忆下协变微商公式
  {\blue
  \bea
  \left(T^{\mu_1\mu_2\ldots}_{\ \ \nu_1\nu_2\ldots}\right)_{;\lambda} &=& \left(T^{\mu_1\mu_2\ldots}_{\ \ \nu_1\nu_2\ldots}\right)_{,\lambda} \newl
  && + \Gamma^{\mu_1}_{\ \rho\lambda} T^{\rho\mu_2\ldots}_{\ \ \nu_1\nu_2\ldots} + \Gamma^{\mu_2}_{\ \rho\lambda} T^{\mu_1\rho\ldots}_{\ \ \nu_1\nu_2\ldots} + \ldots \newl
  && - \Gamma^\rho_{\nu_1\lambda}T^{\mu_1\mu_2\ldots}_{\ \ \rho\nu_2\ldots} - \Gamma^\rho_{\nu_2\lambda}T^{\mu_1\mu_2\ldots}_{\ \ \nu_1\rho\ldots}-\ldots
  \eea
  }
  显然,{\blue 给联络 $\Gamma^\lambda_{\ \mu\nu}$ 加上任何张量 $B^\lambda_{\ \mu\nu}$ 都不会破坏协变微商的保张量性。} 
\end{frame}


\begin{frame}
  现在我们引入度规

  \addfig{1.}{lundaowo.jpg}
\end{frame}


\begin{frame}
  \frametitle{协变微商的物理实体性}
  我们来看 $A^\mu_{\ ;\nu}$ 和 $A_{\mu;\nu}$ 之间的联系。

  \skiplines
  
  在曲面嵌入高维平直空间的假设下,矢量的协变微商具有明确的物理实体对应, $A^\mu_{\ ;\nu}$ 和 $A_{\mu;\nu}$ 对应同一物理实体,所以必然有 $A_{\mu;\nu} = g_{\mu\lambda}A^{\lambda}_{\ ;\nu}$。

  \skiplines

  如果我们进行的是抽象的不依赖于高维平直空间的讨论。那么除了我们恰好用了很具有迷惑性的符号之外,两者其实并无确定的关系。也就是说,如果把 $A_{\mu;\nu}$ 写成 $T_{\mu\nu}$,$A^\mu_{\ ;\nu}$ 写成 $S^\mu_{\ \nu}$,则 $T$ 和 $S$ 虽然都是张量(这由联络的坐标变换规则和协变微商的计算规则即可保证),但 $T$ 和 $S$ 未必对应同一物理量。

\end{frame}


\begin{frame}
  \frametitle{协变微商的物理实体性和度量的普适性之间的关系}
  显然,协变微商具有物理实体性(升降指标和求协变微商次序可以交换)和 度量的普适性(度规的协变微商处处为零)是同一件事情。

  \skiplines
  
  例如,如果度规的协变形式的协变微商为零,就有
  $$A_{\mu;\nu} = \left(g_{\mu\lambda}A^\lambda\right)_{;\nu} = g_{\mu\lambda;\nu}A^\lambda + g_{\mu\lambda}A^\lambda_{\ ;\nu}=g_{\mu\lambda}A^\lambda_{\ ;\nu}.$$
\end{frame}

\begin{frame}
  \frametitle{度量的普适性的等价条件}
  由于 $g^\mu_{\ \nu}$ 是常数矩阵,所以其普通微商为零:
  $$g^\mu_{\ \nu;\lambda} = \Gamma^\mu_{\ \alpha\lambda}g^\alpha_{\ \nu} - \Gamma^\alpha_{\ \nu\lambda}g^\mu_{\ \alpha} = \Gamma^\mu_{\ \nu\lambda} - \Gamma^\mu_{\ \nu\lambda} = 0.$$
  即度规的混合形式的协变微商自动为零,并不给出任何额外限制。
\end{frame}


\begin{frame}
  \frametitle{度量的普适性的等价条件(续)}
  度规的协变形式的协变微商为零给出:
  $$g_{\mu\nu;\lambda} = g_{\mu\nu,\lambda}-g_{\alpha\nu}\Gamma^\alpha_{\ \mu\lambda}-g_{\mu\alpha}\Gamma^\alpha_{\ \nu\lambda}=0.$$
  
  如果上述条件得到满足,我们并不需要额外考虑“度规的逆变形式的协变微商为零”这个条件,这是因为 $g^{\mu\nu} = g^\mu_{\ \lambda}g^\nu_{\ \rho} g_{\lambda\rho}$,而三个因子的协变微商均为零。
\end{frame}

\begin{frame}
  \frametitle{Christoffel-Levi-Civita联络}
    如果空间是 $n$ 维的,则 $g_{\mu\nu;\lambda}= 0$ 给出了 $\frac{n^2(n+1)}{2}$ 个限制方程,还不足以确定 $n^3$ 个联络。不过,{\blue 如果假设联络是对称的 $\Gamma^\lambda_{\ \mu\nu}=\Gamma^\lambda_{\ \nu\mu}$},就可以通过直接化简 $g_{\mu\lambda;\nu}+g_{\nu\lambda;\mu}-g_{\mu\nu;\lambda}=0$ 得到
  {\blue  $$\Gamma_{\lambda\mu\nu}\equiv g_{\lambda\alpha}\Gamma^\alpha_{\ \mu\nu}=\frac{1}{2}\left(g_{\lambda\mu,\nu}+g_{\lambda\nu,\mu}-g_{\mu\nu,\lambda}\right);$$}
  它有个你记不住的名字,叫{\blue Christoffel-Levi-Civita联络,简称 Christoffel 联络}。

  \addfig{0.9}{levicivita.png}
\end{frame}


\begin{frame}
  \frametitle{挠还是不挠?}
  在一般的(不假设嵌入高维空间的图景)理论中,可以在联络里加入额外的一个反对称张量,称为{\blue 挠率张量}。这个挠率张量和曲线的挠率同样有“扭曲”的意思,但本质并不相同:前者描述的是由于空间坐标轴的代数不对称性造成的二阶小量修正,后者描述的是普通欧氏空间里的三阶小量修正。

  \addfig{2}{naonaonao.jpg}
  
  也许是基于我们的时空嵌入在一个高维平直空间里的大胆假设,或者纯粹为了减少理论的自由度,爱因斯坦在{\blue 广义相对论里假设不存在挠率张量,只使用 Cristoffel 联络}。
\end{frame}

\begin{frame}
  \frametitle{测地线}
  由于广义相对论使用 Cristoffel 联络,我们之前对测地线方程的推导完全可以照搬到四维时空来,结果同样是
  \tbox{$$\frac{d^2x^\lambda}{ds^2} + \Gamma^\lambda_{\ \mu\nu}\frac{dx^\mu}{ds}\frac{dx^\nu}{ds} = 0.$$}
  如果说广义相对论分为运动学和引力两个部分,那么上面的测地线方程就是运动学的核心方程。
\end{frame}


\begin{frame}
  在下一讲,我们将转入对四维时空的黎曼张量,为讨论广义相对论的第二核心方程——爱因斯坦引力方程做好准备。
\end{frame}

\ech
\end{document}


