\documentclass[CJK,13pt]{beamer}
\input{macros.tex}
\def\courseurl{http://zhiqihuang.top/gr}

\def\tpage#1#2{
\title{GR \S{#1}  #2}
  \author{Zhiqi Huang}

\begin{frame}
\begin{center}
{\bf \Huge G}eneral {\bf \Huge R}elativity

{\vskip 0.1in}



{\Large \S #1 #2}

{\vskip 0.2in}

{Lecturer: 黄志琦}

\vskip 0.2in

\courseurl

\end{center}
\end{frame}
}


  \date{}
  \begin{document}
  \bch
  \tpage{21}{Physics of Gravitational Waves}

  \begin{frame}
    \frametitle{专业版引力波三连锤}
    上一讲介绍了科普版引力波三连锤。这一讲我们介绍专业版:
    \bitem
  \item{引力(波)只有2个物理自由度,神马意思?}
  \item{引力波是横向无迹的,解释下?}        
  \item{为什么引力波自旋是2?}
    \eitem

    \addfig{3}{shanchu.jpg}
  \end{frame}
  
  \secpage{引力(波)的物理自由度?}{只有2个}
  
  \begin{frame}
    \frametitle{引力波是物理的吗?}
    我们约定了谐和坐标条件
    $$ \square x^\mu = 0$$
    然后导出了引力波要满足的爱因斯坦方程
    $$ \square h_{\mu\nu} = 0$$
    是不是忍不住要怀疑 $h_{\mu\nu}$ 是人为操作出来的?
  \end{frame}



  \begin{frame}
    注意到谐和坐标系不是唯一的,对 $\square \epsilon^\mu = 0$ (学过电动力学的你丝毫不会怀疑这样的 $\epsilon^\mu$ 有无穷多种解)且 $\epsilon^\mu$ 不超过 $h_{\mu\nu}$ 的量级,那么至少在一阶近似下,坐标变换 $x^\mu \rightarrow x^\mu+\epsilon^\mu$ 仍然给出谐和坐标系。

    \skipline
    
    看来操作的空间很大——

    \skipline
    
    愈发怀疑……
  \end{frame}

  \begin{frame}
    为了简化讨论,我们来考虑一个沿 $z$ 轴传播的平面波解。
    \bitem
  \item{$h_{33}$ 描述的是沿传播方向的度量(即物理长度和$z$坐标之间的关系)的振动,这家伙可以通过操作 $z$ 坐标的定义来消除。}
  \item{虽然这有些不太好想象,但是数学上的对称性容易让你相信操作时间坐标可以干掉 $h_{00}$。}
  \item{虽然引力波并不在 $x$ 和 $y$ 方向上移动,但是操作 $x$ 坐标和 $y$ 坐标还是可以干掉 $h_{13}$ 和 $h_{23}$(这类似于动态地调节过$z$轴上各点的 $x$, $y$ 方向和 $z$轴的夹角来消除度规交叉项)。}
    \eitem
  \end{frame}

  \begin{frame}
    注意傅立叶波矢  $k^\mu= (\omega, 0, 0, \omega)$, $k_\mu = (\omega, 0, 0, -\omega)$,谐和坐标条件
    $$\eta^{\mu\nu}h_{\mu\alpha,\nu} = \frac{1}{2}h_{,\alpha}$$
    要求
    $$ k^\mu h_{\mu\alpha} = \frac{1}{2}k_\alpha h$$
    令 $\alpha=0,1,2,3$,分别得到
    $$  h_{30} = \frac{1}{2}h, h_{01}=0, h_{02}=0, h_{03}=-\frac{1}{2}h $$
    所以,$h_{01},h_{02},h_{03}$也自动消失了,且 $h_{11}+h_{22}=0$。

    \skipline
    
     这时候 $h_{\mu\nu}$ 的迹 $h=0$,且满足 $k^\mu h_{\mu\alpha} = 0$,这叫做{\blue 横向无迹(Transverse-Traceless,简称TT)规范}。
  \end{frame}


  \begin{frame}

    最后的结论是: {\blue 对沿 $z$ 轴方向传播的平面波,只有 $h_{11}-h_{22}$ 和 $h_{12}$ 是物理的两个自由度。}其余所有分量以及 $h_{11}+h_{22}$ 都是零。

    \skiplines

  \end{frame}

  \begin{frame}
    
    度规有10个自由度,四个坐标选择的随意性干掉了4个非物理自由度。那么度规应该还有6个物理自由度才对。

    \skiplines
    
    有时候我们会说,引力(没有波字!)只有两个物理自由度。这是什么意思呢?

    \skiplines

    因为引力波可以脱离物质自己传播,所以对应引力的内禀自由度。而度规的其他4个物理自由度由类似于牛顿引力的泊松方程描述(幼儿版GR就不进行这些繁琐的数学推导了),拿掉物质的源,响应就会消失。所以可以认为度规的其他物理自由度其实是物质的自由度,而不是引力本身的。
  \end{frame}

  \secpage{引力子}{是自旋为2的粒子}

  \begin{frame}
    我们经常在科普读物中看到介绍说引力子的自旋为$2$,这是啥意思呢?
  \end{frame}

  \begin{frame}
    我们都知道平面上自旋为 $1$ 的,也就是矢量场 $\vecv$ 在坐标系旋转 $\theta$  角时,满足下列变换规则:
    \be
      \tilde{v}_x = v_x\cos{\theta} + v_y\sin{\theta},\ \tilde{v}_y = v_y\cos{\theta} - v_x\sin{\theta}.
    \ee
    自旋为2的场的变换规则很简单,就是把 $\theta$ 变为 $2\theta$。
    
    \addfig{1.}{chulecai.jpg}

    但是——这好像有些超出本喵的想象力
  \end{frame}
  
  
  \begin{frame}
    为了获得一些直观理解,我们先来看一些由$x$-$y$平面上的标量场构造出来的自旋为 $2$ 的场。设 $E$ 是一个标量场,定义
    \be
      q_E \equiv \frac{1}{2}\left(\partial_x^2 - \partial_y^2\right)E,\ u_E = \partial_x\partial_yE.
    \ee
  \end{frame}

  \begin{frame}
    当$x$-$y$坐标系绕原点旋转$\theta$至 $\tilde{x}$-$\tilde{y}$坐标系时,
    \be
      \frac{\partial x}{\partial\tilde{x}} = \cos\theta,\ \frac{\partial x}{\partial \tilde{y}} = -\sin\theta, \ \frac{\partial y}{\partial\tilde{x}} = \sin\theta,\ \frac{\partial y}{\partial \tilde{y}} = \cos\theta, \ 
      \ee
    于是有
    \be
      \frac{\partial E}{\partial\tilde{x}} = \frac{\partial E}{\partial x} \cos\theta + \frac{\partial E}{\partial y}\sin\theta,\ \frac{\partial E}{\partial \tilde{y}} = \frac{\partial E}{\partial y}\cos\theta-  \frac{\partial E}{\partial x} \sin\theta.
    \ee      
  \end{frame}

  \begin{frame}
    以及
    \begin{eqnarray}
      \frac{\partial^2 E}{\partial\tilde{x}^2} &=& \frac{\partial^2 E}{\partial x^2} \cos^2\theta + \frac{\partial^2 E}{\partial y^2}\sin^2\theta + \frac{\partial^2 E}{\partial x \partial y}\sin 2\theta, \\
      \frac{\partial^2 E}{\partial\tilde{x}\partial\tilde{y}} &=& \frac{\partial^2E}{\partial x \partial y} \cos 2\theta  - \frac{1}{2}\left(\frac{\partial^2 E}{\partial x^2}-\frac{\partial^2 E}{\partial y^2}\right)\sin 2\theta, \\  
      \frac{\partial^2 E}{\partial\tilde{y}^2} &=& \frac{\partial^2E}{\partial y^2} \cos^2\theta + \frac{\partial^2E}{\partial x^2}\sin^2\theta - \frac{\partial^2E}{\partial x \partial y}\sin{2\theta},   
    \end{eqnarray}
    或者写成更紧凑的形式
    \be
      \tilde{q}_E = q_E\cos{2\theta} + u_E\sin{2\theta},\ \tilde{u}_E = u_E\cos{2\theta} - q_E\sin{2\theta}.
    \ee
    这说明我们从标量场 $E$ 构造的 $(q_E, u_E)$ 是自旋为 $2$ 的场。
  \end{frame}

  \begin{frame}
    对于沿 $z$ 轴方向传播($x,y$坐标均为零)的引力波,假设我们已经操作掉了其他自由度,只剩下 $h_{11}-h_{22}$ 和 $h_{12}$。

      对局域的“标量”函数
      $$ E(x,y) = \frac{1}{2}h_{11}x^2 +  \frac{1}{2}h_{22}y^2 + h_{12}xy $$
      (它只是局部的 $\frac{1}{2}ds^2$ 而已)
        计算 $q_E, u_E$, 即可知道 $\left(\frac{h_{11}-h_{22}}{2}, h_{12}\right)$ 是自旋为2的场。
        在文献中,一般不写 $q_E, u_E$,而是用符号
        $$h_+ \equiv \frac{h_{11}-h_{22}}{2}, h_\times \equiv h_{12}$$
        来表示引力波的两种模式。
  \end{frame}

  \begin{frame}
    \addfig{3}{hpluscross.png}
  \end{frame}

  
  \begin{frame}
    要注意的是,$ds^2$ 不能是全局的标量,因此引力波(微扰意义下的二阶张量)不能用一个全局的标量场来描述。
  \end{frame}

  
    \ech
\end{document}




  
