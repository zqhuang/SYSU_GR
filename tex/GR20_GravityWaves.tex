\documentclass[CJK,13pt]{beamer}
\input{macros.tex}
\def\courseurl{http://zhiqihuang.top/gr}

\def\tpage#1#2{
\title{GR \S{#1}  #2}
  \author{Zhiqi Huang}

\begin{frame}
\begin{center}
{\bf \Huge G}eneral {\bf \Huge R}elativity

{\vskip 0.1in}



{\Large \S #1 #2}

{\vskip 0.2in}

{Lecturer: 黄志琦}

\vskip 0.2in

\courseurl

\end{center}
\end{frame}
}


  \date{}
  \begin{document}
  \bch
  \tpage{20}{Gravitational Waves}

  \begin{frame}

    往水里扔一块石头,水的流体元的空间位置随着时间波动性地变化。

    \addfig{2}{waterwave.jpg}

    可以这样说:水波是 \uline{水} 相对于 \uline{空间背景} 的 \uline{时间}周期性 振动。

  \end{frame}

  
  \begin{frame}
    引力波通常在科普读物中被描述为“时空的涟漪”
    
    \addfig{2}{yinlibo.jpg}

    那引力波是 \uline{时空} 相对于 \uline{???} 的 \uline{??} 周期性 振动?

  \end{frame}

    \ech
\end{document}




  
