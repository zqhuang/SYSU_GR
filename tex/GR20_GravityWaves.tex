\documentclass[CJK,13pt]{beamer}
\input{macros.tex}
\def\courseurl{http://zhiqihuang.top/gr}

\def\tpage#1#2{
\title{GR \S{#1}  #2}
  \author{Zhiqi Huang}

\begin{frame}
\begin{center}
{\bf \Huge G}eneral {\bf \Huge R}elativity

{\vskip 0.1in}



{\Large \S #1 #2}

{\vskip 0.2in}

{Lecturer: 黄志琦}

\vskip 0.2in

\courseurl

\end{center}
\end{frame}
}


  \date{}
  \begin{document}
  \bch
  \tpage{20}{Gravitational Waves}


  \begin{frame}
    往水里扔一块石头,水的流体元的空间位置随着时间波动性地变化。

    \addfig{1.5}{waterwave.jpg}

    可以这样说:水波是 \uline{水} 相对于 \uline{空间背景} 的 \uline{时间}周期性 振动。
  \end{frame}


  \begin{frame}
    \frametitle{问题I}
    引力波通常在科普读物中被描述为“时空的涟漪”
    
    \addfig{1.5}{yinlibo.jpg}

    那引力波是 \uline{??} 相对于 \uline{???} 的 \uline{??} 周期性 振动?
  \end{frame}


  \begin{frame}
    \frametitle{问题II}
    爱因斯坦方程
    $$R^{\mu\nu}-\frac{1}{2}Rg^{\mu\nu} = 8\pi G T^{\mu\nu}$$
    这里的 $T^{\mu\nu}$ 要包含引力波的能量动量吗?
  \end{frame}


  \begin{frame}
    \frametitle{问题III}
    引力波经过引力源附近也会发生偏折吗?
  \end{frame}


  \begin{frame}
    谁说知道引力波是啥,你就拿这三连发锤TA。

    \addfig{0.8}{lahei.jpg}
  \end{frame}

  \secpage{谐和坐标系的引力波方程}{$$\square h_{\mu\nu} = -16\pi G\left(T_{\mu\nu}- \frac{1}{2}\eta_{\mu\nu}T\right)$$}

    \begin{frame}
    \frametitle{四维拉普拉斯算符$\square$}
    四维拉普拉斯算符 $\square$ 的定义为
    $$\square f = f^{;\mu}_{\ \ ;\mu}$$
    对一个标量 $\phi$,有
    \bea
    \square \phi &=& \left(g^{\mu\nu}\phi_{;\nu}\right)_{;\mu} \newl
    &=& g^{\mu\nu}\phi_{;\nu;\mu} \newl
    &=& g^{\mu\nu}\phi_{,\nu,\mu} - g^{\mu\nu}\Gamma^\lambda_{\ \nu\mu}\phi_{,\lambda} \newl
    &=& g^{\mu\nu}\phi_{,\nu,\mu} - \Gamma^\lambda\phi_{,\lambda} 
    \eea
    在最后一行我们定义了 $\Gamma^\lambda\equiv g^{\mu\nu}\Gamma^\lambda_{\mu\nu}$
  \end{frame}


    \begin{frame}

      
      显然,Minkowski时空的里相速度为光速的波动方程可以写成 $\square u = 0$.

      \addfig{1}{haowubodong.jpg}


      
  \end{frame}
    
  
  \begin{frame}
    \frametitle{近似Minkowski时空}
    如果观测者远离中子星和黑洞这样稀少的天体。我们知道广义相对论效应都不强,时空近似是 Minkowski 时空。

    \skipline
    
    不仅如此,爱因斯坦方程右边的 $8\pi G T^{\mu\nu}$ 的量级可以用白矮星的密度来设置一个上限:
    $$\lvert 8\pi G T^{\mu\nu}\rvert \lesssim \frac{1}{L^2}$$
    这里的 $L=10^9\mathrm{m}$。由于爱因斯坦张量是度规的二阶导数以及一阶导数的平方组成的,所以就可以认为在小于 $L$ 的尺度范围内度规的变化量也很小。这当然不能算是严格的证明,但至少是作为推理出发点的合理假设。
  \end{frame}


  \begin{frame}
    \frametitle{近Minkowski度规}
    因此,我们假设在远离致密天体的小范围时空区域里,{\bf 选择合理的坐标系},可以有度规
    $$ g_{\mu\nu} = \eta_{\mu\nu}+h_{\mu\nu}$$
    这里的 $\eta_{\mu\nu}$ 是Minkowski度规,$h_{\mu\nu}\ll 1$,且 $h_{\mu\nu}$ 的一阶、二阶偏导数都可以看成同阶的小量。我们将在讨论运动方程时总是{\blue 只保留 $h_{\mu\nu}$ 的一阶小量}。
  \end{frame}


  \begin{frame}
    \frametitle{“合理坐标系”不清不楚,不如叫“\sout{河蟹}谐和坐标系”}
    我们要求“谐和坐标系”对所有指标$\mu=0,1,2,3$,都要满足
  
    \addfig{0.8}{crab.jpg}
  
    取标量 $f = x^\mu$ (那就是说,即使你换到另外一种不太河蟹的坐标系,每个物理点的 $f$ 还是原先谐和坐标系里的 $x^\mu$),都要有
    $$\square f = 0$$
  \end{frame}

  \begin{frame}
    \frametitle{谐和坐标条件(续)}
    根据
    $$ \square f = g^{\mu\nu}f_{,\mu,\nu} - \Gamma^\lambda f_{,\lambda}$$
    在谐和坐标系里, $f=x^\mu$ 的任意二阶普通偏导都是零,所以谐和坐标条件可以等价地写为
    $$\Gamma^\lambda = 0$$
    当然,也可以写成
    $$\Gamma_\lambda \equiv g^{\alpha\beta}\Gamma_{\lambda\alpha\beta}=0$$
  \end{frame}




  \begin{frame}
    \frametitle{线性化的谐和坐标条件}
    我们先计算联络
    $$\Gamma_{\rho\mu\nu} = \frac{1}{2}\left(h_{\mu\rho,\nu}+h_{\nu\rho,\mu}-h_{\mu\nu,\rho}\right)$$    
    谐和坐标条件就是
    $$\frac{1}{2}\eta^{\mu\nu}\left(h_{\mu\rho,\nu}+h_{\nu\rho,\mu}-h_{\mu\nu,\rho}\right) = 0$$
    上式可以写成
    $$ \eta^{\mu\nu}h_{\rho\mu,\nu} =\frac{1}{2} h_{,\rho} $$
    这里的 $h\equiv \eta^{\mu\nu}h_{\mu\nu}$.
  \end{frame}


  \begin{frame}
    \frametitle{谐和坐标系里的Ricci张量一阶近似}
    由于联络已经是 $h_{\mu\nu}$ 的一阶小量,在黎曼张量的联络表达式里就可以扔掉联络的乘积项,
    {\small
    \bea
    R_{\mu\nu} &=& R^\lambda_{\ \mu\nu\lambda} \newl
    &=& \Gamma^\lambda_{\ \mu\nu,\lambda} - \Gamma^\lambda_{\ \mu\lambda,\nu} \newl
    &=&  \frac{1}{2}\eta^{\lambda\rho}\left(h_{\mu\rho,\nu,\lambda}+h_{\nu\rho,\mu,\lambda}-h_{\mu\nu,\rho,\lambda}+h_{\rho\mu,\lambda,\nu}+h_{\rho\lambda,\mu,\nu}-h_{\mu\lambda,\rho,\nu}\right) \newl
    &=&  \frac{1}{2}\eta^{\lambda\rho}\left(h_{\nu\rho,\mu,\lambda}+h_{\rho\mu,\lambda,\nu}-h_{\mu\nu,\rho,\lambda}\right) + \frac{1}{2}h_{,\mu,\nu}     \newl
    &=&  -\frac{1}{2}\eta^{\lambda\rho} h_{\mu\nu,\rho,\lambda}     \newl
    \eea
    }
    在最后一步我们使用了谐和坐标条件        
  \end{frame}


  \begin{frame}
    \frametitle{真空中的爱因斯坦方程:引力波}
    于是在谐和坐标系下,真空中的爱因斯坦方程可以写作:
    $$ \square h_{\mu\nu} = 0$$
    这是波速为光速的波动方程。
  \end{frame}


  \begin{frame}
    \frametitle{非真空情况}
    在非真空中,谐和坐标系里线性化的爱因斯坦方程可以写作:{\blue
      $$\square h_{\mu\nu} = S_{\mu\nu}$$}
    这里的
    $$ S_{\mu\nu} \equiv -16\pi G\left(T_{\mu\nu} - \frac{1}{2}\eta_{\mu\nu}T^\alpha_{\ \alpha}\right)$$
    为了让解看起来更幼儿园化,下面我们把时间和空间坐标分离开来,记 $t=x^0$, $\mathbf{x} = (x^1, x^2, x^3)$。    
  \end{frame}

  \begin{frame}
    \frametitle{引力辐射方程的解}
    如果你\uline{贫瘠的记忆里还有一丝电动力学的残余物},或许下面的“推迟势”解并不会让你太慌张:
    $$ h_{\mu\nu}(t, \vecx) = \frac{1}{4\pi}\int d^3\vecx' \frac{S_{\mu\nu}\left(\vecx', t - \lvert \vecx -\vecx'\rvert\right)}{\lvert \vecx - \vecx'\rvert} $$

    \addfig{0.8}{wangji.jpg}
    
    当然这是一个特解,你可以把任何真空中的 $\square h_{\mu\nu}=0$ 的解加上去。和电动力学中一样,我们把这个特解解释为由源 $S_{\mu\nu}$ 产生的引力波,而把额外的真空解部分(如果有的话)解释为来自于远处未知源的引力波。
  \end{frame}
  
    \ech
\end{document}




  
