\documentclass[CJK,13pt]{beamer}
\input{macros.tex}
\def\courseurl{http://zhiqihuang.top/gr}

\def\tpage#1#2{
\title{GR \S{#1}  #2}
  \author{Zhiqi Huang}

\begin{frame}
\begin{center}
{\bf \Huge G}eneral {\bf \Huge R}elativity

{\vskip 0.1in}



{\Large \S #1 #2}

{\vskip 0.2in}

{Lecturer: 黄志琦}

\vskip 0.2in

\courseurl

\end{center}
\end{frame}
}


  \date{}
  \begin{document}
  \bch
  \tpage{16}{White Dwarf, Neutron Star and Black Hole}

 

  \begin{frame}
    \frametitle{这一讲的主题是:致密星(白矮星、中子星、黑洞)的形成}

    \addfig{1}{biehuang.jpg}
    
    别慌,我们只做个简单的数量级估计。
    
  \end{frame}


  \begin{frame}
    \frametitle{坍缩前}

    \addfig{1.5}{star.jpg}
    
    恒星主要靠燃烧(聚变)氢来维持高温高压,抵抗自身引力。

    \skiplines

    一般来说,氢管够,够败上亿年……
    
  \end{frame}
  


  \begin{frame}

    不管是几亿年还是几十亿年,扛不住\sout{马云}宇宙爸爸活得久——最后还是败完了……

    \addfig{1.3}{yaofan.jpg}

    在要饭前其实还可能靠典当氦,碳……短暂挣扎了一会儿,我们不纠结这些细节。
    
  \end{frame}


  \begin{frame}

    \addfig{1.5}{yaozhai.jpg}
    
    一旦进入要饭模式,引力大哥就毫不客气地开始讨债……
    
  \end{frame}


  \begin{frame}

    假设恒星质量是 $M$,当它失去了聚变的支撑被引力压缩到半径为 $R$ 时——反正都完蛋了,我们来数数吧:

    \lfig{1}{sheep.jpg}  \lfig{1}{sheep.jpg}  \lfig{1}{sheep.jpg}

    根据泡利费米子要打架原理,恒星里的大概 $M/m_p$ ($m_p$ 是质子质量)个电子,每个占据了一个微观态。根据海森堡微观态也要打架原理,每个微观态在相空间占据体积为$h^3$——那么电子在动量空间的占据体积为

    $$ p_{\max}^3 \sim \frac{Mh^3}{m_pR^3} $$
    
    {\scriptsize 我们忽略了$\frac{4\pi}{3}$、自旋自由度可以干掉一个因子2以及电子个数其实比 $M/m_p$ 还要少些等无关数量级的因素。}
  \end{frame}


  \begin{frame}
    如果动量是非相对论性的,压强(单位面积单位时间通过动量)大致为
    $$P \sim p_{\max} \left(\frac{M}{m_pR^3} \frac{p_{\max}}{m_e}\right)\sim \left(\frac{M}{m_p}\right)^{5/3}\frac{h^2}{R^5m_e}$$
    上式能成立的条件是 $\frac{M}{R^3} \ll \frac{m_e^3c^3m_p}{h^3}$。

    {\scriptsize 极端相对论的情况无法有稳态的理由并不复杂,不过我们跳过这个细节……}

    \skipline

    压力和引力之比

    $$ \sim \frac{PR^2}{GM^2/R^2} \sim \frac{h^2}{GM^{1/3}m_p^{5/3}m_eR} $$

    如果上式能取到 $\sim 1$,将代表稳定解:坍缩会导致 $R$ 减小,压力超过引力。膨胀会导致 $R$ 增大,引力超过压力。
    
  \end{frame}

  
  \begin{frame}
    同时要求 $\frac{h^2}{GM^{1/3}m_p^{5/3}m_eR} \sim 1$ 且 $\frac{M}{R^3} \ll \frac{m_e^3c^3m_p}{h^3}$ 等价于要求

    $$ M \ll \frac{1}{m_p^2}\left(\frac{hc}{G}\right)^{3/2}= 29M_\odot$$

    你当然不能指望这么粗糙的估算能得到Chandrasekhar质量极限 ($M<1.4M_\odot$),不过,数量级已经很靠谱了。

    \skiplines
    
    对一个太阳质量的白矮星,稳定半径
    $$ R\sim \frac{h^2}{GM_\odot^{1/3}m_p^{5/3}m_e} \sim 10^5\mathrm{km}$$
    同样,这个数大致OK,但更准确的结果还要小一个数量级(白矮星大致为地球大小)。
    
  \end{frame}
  

  \begin{frame}
    等一下,质量极限的结果好像和电子质量无关?那么中子星的质量极限岂不是……
  \end{frame}


  \begin{frame}
    没错,中子星的质量极限跟白矮星的相差无几!

    \skiplines
    
    差别在于计算稳定半径时要除以中子质量而非电子质量,所以中子星的稳定半径要比白矮星小三个数量级左右($10$公里的量级)。
  \end{frame}


  \thinkf{典型的中子星内部压力比白矮星高多少个量级?}  


  \begin{frame}
    \frametitle{黑洞——大质量恒星的宿命}
    如果恒星尸体的质量超过若干个太阳质量,那么无论是白矮星、中子星还是尚有争论的夸克星,都无法稳定存在。也就是费米压的抵抗引力的机制失效了。

    \skiplines

    这时一般认为没有什么可以阻止恒星尸体坍缩为时空奇点——黑洞。
  \end{frame}


  \begin{frame}
    \frametitle{史瓦西黑洞}
    $$ ds^2 = \left(1-\frac{2GM}{r}\right)dt^2 - \left(1-\frac{2GM}{r}\right)^{-1} dr^2 - r^2\left(d\theta^2 + \sin^2\theta d\phi^2\right)$$
    {\scriptsize 史瓦西黑洞是比较理想化的黑洞,实际的黑洞由于吸积了周围的物质,都是有角动量的克尔黑洞。我们在hard模式再讨论……}
  \end{frame}


  \begin{frame}
    \frametitle{黑洞照片}
    \addfig{2}{EHT_BH.jpg}
  \end{frame}
  
\ech
\end{document}




  
