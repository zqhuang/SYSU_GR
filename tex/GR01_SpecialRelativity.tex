\documentclass[CJK,13pt]{beamer}
\input{macros.tex}
\def\courseurl{http://zhiqihuang.top/gr}

\def\tpage#1#2{
\title{GR \S{#1}  #2}
  \author{Zhiqi Huang}

\begin{frame}
\begin{center}
{\bf \Huge G}eneral {\bf \Huge R}elativity

{\vskip 0.1in}



{\Large \S #1 #2}

{\vskip 0.2in}

{Lecturer: 黄志琦}

\vskip 0.2in

\courseurl

\end{center}
\end{frame}
}


  \date{}
  \begin{document}
  \bch
\tpage{1}{Special Relativity}


\begin{frame}
  \frametitle{我们都知道勾股定理}
  \addfig{4}{gougu.png}
\end{frame}


\begin{frame}
  \frametitle{还有二维直角坐标系的坐标变换}
  \addfig{2}{coor2Dtrans.jpg}
\end{frame}


\begin{frame}
  \frametitle{由这些可以推导出洛仑兹变换}
  \begin{eqnarray}
  t'&=&\frac{t-vx/c^2}{\sqrt{1-v^2/c^2}}; \newl
  x'&=&\frac{x-vt}{\sqrt{1-v^2/c^2}}; \newl
  y'&=&y; \newl
  z'&=&z. \nonumber
  \end{eqnarray}
\end{frame}


\begin{frame}
  \frametitle{老写一堆$c$太麻烦了,用自然单位制吧}
  我们完全可以用``米''做为时间单位: 1米时间就是光在真空中走过1米所需要的时间。
  \begin{eqnarray}
  t'&=&\frac{t-vx}{\sqrt{1-v^2}}; \newl
  x'&=&\frac{x-vt}{\sqrt{1-v^2}}; \newl
  y'&=&y; \newl
  z'&=&z. \nonumber
  \end{eqnarray}
\end{frame}

\begin{frame}
  \frametitle{事件和世界线}
  \bitem
  \item{在某时某地发生的{\bf 事件}可以用四维坐标来表示。事件是一个物理存在,和描述它的坐标系无关。两个事件之间的``距离平方'' $$s^2=\Delta t^2 - \Delta x^2 - \Delta y^2 - \Delta z^2,$$同样不依赖于坐标系的选取。}
  \item{在某时某地有一个可以抽象为质点的物理对象(例如,一个电子),是一个事件。如果这个物理对象持续地存在于时空中,那么可以表述为``它存在于某时某地''的所有事件在四维时空中留下了一条连续的“轨迹”,称为这个物理对象的“{\bf 世界线}”。世界线同样是一个物理存在,和描述它的坐标系无关。}
    \eitem
\end{frame}




\begin{frame}
  \frametitle{学习狭义相对论的正确姿势}
  \bitem
\item{理解狭义相对论的惯性系,(惯性)观测者,以及当我们说“观测者看到xxx”是什么意思。把所有描述精确地翻译为四维概念。}
\item{用洛仑兹变换和不变量进行计算。当涉及力的时候,不要用  $F=ma$(狭义相对论范畴内不正确),要用 $\frac{dp}{dt} = F$(狭义相对论范畴内仍然正确)。注意粒子的动量为
  $\frac{mv}{\sqrt{1-v^2}}$。同样,当用能量守恒定律时要注意粒子的总能量为 $\frac{m}{\sqrt{1-v^2}}$ (而不是 $m+\frac{mv^2}{2}$)。}
\item{当涉及电场、磁场等时,如果你想继续使用幼儿园的电磁学,不想用电动力学里的四维矢势,那你要注意哪些熟悉的“初等公式”仍然是适用的:电场力 $\mathbf{F}=q\mathbf{E}$,洛仑兹力
$\mathbf{F} = q\mathbf{\upsilon}\times \mathbf{B}$ 都可以放心使用,但是运动电荷产生的电场——算了劝你还是上电动力学吧。}
  \eitem

\end{frame}


\begin{frame}
  \frametitle{我们说观测者看到动尺变短是什么意思?}
  
   \addfig{3}{dongchi.jpg}
  
\end{frame}


\begin{frame}
  \frametitle{我们说观测者看到运动的时钟变慢是什么意思?}
  
  \addfig{3}{time_dilation.jpg}
\end{frame}


\begin{frame}
  \frametitle{观测者看到……}
  
  
  你能总结下狭义相对论里的“观测者看到……”具体是什么意思吗?
  
\end{frame}


\begin{frame}
  \frametitle{狭义相对论的语言陷阱}
  {\small
  \bitem
\item{狭义相对论假设惯性系以及惯性系的“上帝视角”存在。如果不另加说明,说“观测者看到...事件”通常默认是指...事件在该观测者的固连参考系的四维坐标。这些坐标是以“上帝视角”(假想全宇宙各处都安装了和观测者相对静止的时钟和刻度尺)看到的,而不是观测者用眼睛或者耳朵或者任何仪器测量到的具体信号。这时计算比较简单,只要严格把语言翻译成参考系的坐标,不同参考系的坐标之间的变换由洛仑兹变换公式给出。大多数教材中描述的“动尺变短,动钟变慢”等都是这种“上帝视角”观测量。}  
\item{如果问题明确指出了观测者以某种手段看到某种信号,则要小心了:这时讨论的不再是“上帝视角”观测量。一个简单的例子就是:如果你在地球上用望远镜观测一个远离你的动钟,除了一般教材上描述的“上帝视角”的动钟变慢效应,它发出的光信号到地球所需的时间越来越长也会造成一个看起来钟走得更慢的效应。这两个效应叠加,会造成“动钟看起来比教材说得还要慢”的结果。此外,多普勒效应讨论的也是实际观测量——所以多普勒效应公式和“动尺变短”、“动钟变慢”的公式不大一样。}
  \eitem
  }
\end{frame}


\thinka{一个速度为 $v=\frac{5}{13}$ 的航天旅行者和他的地球上的朋友在出发时对好了钟的时刻为 $t'=0$ (旅行者的时钟) 和 $t=0$ (地球上的钟)。地球上的朋友同时观察两个钟,直接观察 $t$,用望远镜观察 $t'$。当 看到 $t'$ 读数为 $1\mathrm{h}$ 时, 看到 $t$ 的读数为多少h? }


\thinkb{考虑一个非常遥远的距离太阳系为 $L$ ($L$超过百万光年的量级)的星系。一个还剩下寿命 $T$ ($T$约为几十年的量级)的人,是否可能乘坐匀速飞行的宇宙飞船活着到达这个遥远的星系?如有可能,飞船的速率至少要多大?}




\begin{frame}
  \frametitle{python的符号演算库sympy}
  在本课程中,我们将借助python的符号演算库sympy来进行运算。

  {\darkgreen import sympy as sym}

  {\scriptsize (注:一般不提倡为了省事而 from sympy import * ,这样可能会产生大量的重名冲突问题;你当然也知道用 numpy 时,尽量要采用 import numpy as np 而不是 from numpy import *,都是同一个道理)}

  \skiplines

  sympy可以定义符号变量,例如定义一个变量 $x$

 {\darkgreen x = sym.Symbol('x')}

  \skipline

  或者批量定义一些变量

  {\darkgreen x, y, z = sym.symbols('x, y, z')}

  \skipline
  
  可以化简

  {\darkgreen print(sym.simplify(sym.sqrt(1+x)**2))}
  

\end{frame}

\begin{frame}
  \frametitle{python的符号演算库sympy}
  求极限
  
  {\darkgreen print(sym.limit((sym.log(1+x)-x)/x**2, x, 0))}

  \skipline
  
  求偏导运算:

 {\darkgreen print(sym.diff( y*(x**2+sym.sin(x)), x))}

  {\scriptsize (注意,有些老版本python的print命令不用括号,而是用空格隔开)}

  \skipline

  积分运算

  {\darkgreen  print(sym.integrate(sym.log(x),x))}

  \skipline

  级数展开
  
  {\darkgreen print(sym.series(sym.cos(x),x, 0, 12))}


  \skipline
  
  我们的热身练习是用sympy进行洛仑兹变换,请参考

  \url{http://zhiqihuang.top/gr/codes/lorentz_trans.py}
\end{frame}

\begin{frame}
  \frametitle{速度合成}
  用sympy我们很轻松地推导出了速度合成公式:假设惯性参考系 $K'$ 相对于惯性参考系 $K$以沿 $x$ 轴方向的速度 $u$ 运动,且两个坐标系的 $x,y,z$ 坐标轴方向均重合。某物体在 $K'$ 系里以速度 $\vecv = (\upsilon_x,\upsilon_y, \upsilon_z)$ 运动。那么在 $K$ 参考系里看到该物体的速度为:

  {\blue
    \be
    (\frac{\upsilon_x+u}{1+u\upsilon_x}, \frac{\upsilon_y\sqrt{1-u^2}}{1+u\upsilon_x}, \frac{\upsilon_z\sqrt{1-u^2}}{1+u\upsilon_x})
    \ee
    }
  
\end{frame}


\begin{frame}
  \frametitle{孪生子佯谬}
  \addfig{3}{twin_paradox.jpg}
\end{frame}


\ech
\end{document}
