\documentclass[CJK,13pt]{beamer}
\input{macros.tex}
\def\courseurl{http://zhiqihuang.top/gr}

\def\tpage#1#2{
\title{GR \S{#1}  #2}
  \author{Zhiqi Huang}

\begin{frame}
\begin{center}
{\bf \Huge G}eneral {\bf \Huge R}elativity

{\vskip 0.1in}



{\Large \S #1 #2}

{\vskip 0.2in}

{Lecturer: 黄志琦}

\vskip 0.2in

\courseurl

\end{center}
\end{frame}
}


  \date{}
  \begin{document}
  \bch
\tpage{5}{Connection}



\begin{frame}
  回忆一下上一讲的结果: 设三维欧氏空间里的曲面 $\vecx(u^1,u^2)$ 有第一基本形式
  $$ds^2 = g_{ij}du^idu^j,$$
  和第二基本形式
  $$\vecn\cdot d^2\vecx = \beta_{ij}du^i du^j,$$  
  这里的 $\vecn$ 是法向量, $g_{ij} = \vecx_{,i}\cdot\vecx_{,j}$, $\beta_{ij} = \vecn \cdot \vecx_{,i,j} = -\vecn_{,i} \cdot \vecx_{,j} = -\vecx_{,i} \cdot \vecn_{,j}$。

  
  本讲的任务是用第一基本形式(度规 $g_{ij}$)计算高斯曲率 $K = \frac{\det(\beta)}{\det(g)}$。

  \skiplines
  
  在此之前,我们又要引入一些新的符号:请准备好呼吸机……
\end{frame}

\begin{frame}
  \frametitle{度规和指标的升降}
 第一基本形式里的度规矩阵
  \be
  \begin{pmatrix}
    g_{11} & g_{12} \\
    g_{21} & g_{22} 
  \end{pmatrix}
  \ee
  由高维欧式空间的距离平方导出,所以一定是正定矩阵。但是在更一般的讨论中,我们将放宽这一假设,只要求它是满秩矩阵。

  不管怎么说,度规矩阵的逆矩阵存在,我们把它写成
  \be
  \begin{pmatrix}
    g^{11} & g^{12} \\
    g^{21} & g^{22} 
  \end{pmatrix}
  \equiv   \begin{pmatrix}
    g_{11} & g_{12} \\
    g_{21} & g_{22} 
  \end{pmatrix}^{-1}
  \ee
  度规矩阵和它的逆的乘积当然是单位矩阵,这一事实用爱因斯坦的求和约定写出来就是
  \be
  g_{ij}g^{jk} = \delta_i^{\ k} \equiv \left\{
  \begin{matrix}
    1, & \text{if } i = k \\
    0, & \text{else}
  \end{matrix}
  \right.
  \ee
\end{frame}


\begin{frame}
  \frametitle{指标的升降}
  我们搞那么多事情,一会儿把指标写上面(例如 $du^i$),一会儿把指标写下面(例如 $g_{ij}$),是因为上下指标含义不同,且可以互相变换。

  \skipline
  
 具体来说,我们将遵循这样的原则对带指标的量进行“指标升降”。
  \bitem
\item{如果要把一个上指标 $\alpha$ 降为下指标,我们把上指标里的 $\alpha$ 换成任何一个新的没有用到过的指标,例如 $\beta$,然后乘以 $g_{\beta\alpha}$。}
\item{如果要把一个下指标 $\alpha$ 升为上指标,我们把下指标里的 $\alpha$ 换成任何一个新的没有用到过的指标,例如 $\beta$,然后乘以 $g^{\beta\alpha}$。}
  \eitem
  当然,上面规则中的$\beta$ 这个指标是什么不重要,因为按爱因斯坦求和规则我们会对 $\beta$ 进行遍历求和。  
\end{frame}


\begin{frame}
  例如,我们可以把 $du^i$ 里的上指标 $i$ 降下去,具体就是:
  $$ du_i \equiv g_{ij} du^j.$$
  粗略地看,这无非就是个定义而已。

  \skipline

  问题是,这样的定义是自洽的吗?如果我们把 $du_i$ 里的下指标再升上去,还会是和原来一样的 $du^i$ 吗?
\end{frame}



\begin{frame}
  答案是肯定的:

  $$ g^{ij}du_j = g^{ij} g_{jk} du^k = \delta^i_{\ k}du^k = du^i.$$

  很容易把这个简短的证明推广到任意多个指标的任意升降。
\end{frame}


\thinkf{$g^{ij}$ 比较特殊,它是通过 $g_{ij}$矩阵的逆矩阵定义的。请检验,
  \bitem
\item{如果把 $g_{ij}$ 的指标按照上述升降指标的规则升上去,会得到(按逆矩阵定义的) $g^{ij}$ 吗? }
\item{仅上升一个指标的度规 $g_i^{\ j}$ 或者 $g^i_{\ j}$ 是什么?}  
  \eitem}


\begin{frame}
  \frametitle{指标升降操和偏导操作不可交换}

  对坐标的偏导会产生额外的指标。要格外小心的是,指标的升降操作和偏导产生新指标的操作一般是不可交换的。例如: 我们可以把 $g_{ij}$ 的指标 $i$ 升上去得到 $g^i_{\ j}$,然后求偏导得到 $g^i_{\ j, k}$。如果我们先求偏导,得到 $g_{ij, k}$ ,然后把指标 $i$ 升上去,会得到一个数值上不一样的结果。

  \skiplines

  在我们之后学习了张量和协变导数之后,这个问题会得到解决(协变导数和指标升降的操作可以交换)。目前我们对带着偏导的量的指标还是要十分小心,不要随便进行指标升降。

\end{frame}


\begin{frame}
  \frametitle{小心地应用指标升降的例子}
  \addfig{0.8}{eg.jpg}
  
  虽然 $\vecx_{,i}$ 带着偏导指标,我们还是可以强行定义 $\vecx^{,i}=g^{ij}\vecx_{,j}$。由于 $\vecx^{,i}$ 并没有其他什么定义方式,这种操作还不至于导致混乱。

  \skiplines
  
  容易严格按照定义验证
  \be
  \vecx_{,i}\cdot \vecx^{,j} = g_i^{\ j}=\left\{
  \begin{array}{ll}    1, & \text{ if } i=j \\
    0, & \text{ else}
  \end{array}
  \right.
  \ee
  
  $$\vecx^{,i}\vecx^{,j} = g^{ij}$$

\end{frame}




\begin{frame}
  好了,摘下呼吸机。我们要开始操作了。
\end{frame}


\begin{frame}  

  \frametitle{秘密武器}
  我们先来介绍一个秘密武器:{\blue  三维欧氏空间任意矢量 $\vecv$ 可以分解成
  $$ \vecv = \left(\vecx_{,i}\cdot\vecv\right)\vecx^{,i}  + (\vecn\cdot\vecv)\vecn.$$}
  这个等式是这样证明的,在两边点乘 $\vecx_{,j}$ 或 $\vecn$ 都得到相同的结果(这里需要用到 $\vecx_{,i}\cdot\vecx^{,j} = g_i^{\ j}$)。由于 $\vecx_{,1}$, $\vecx_{,2}$, $\vecn$ 是完备的,故得证。

  \addfig{1}{caozuo.jpg}

\end{frame}

\begin{frame}
  \frametitle{联络(connection)}
  我们定义一个非常有用的辅助记号: {\blue 联络
    $$\Gamma_{kij} \equiv \vecx_{,k}\cdot\vecx_{,i,j}.$$}
  显然这样定义的联络具有 $\Gamma_{kij} = \Gamma_{kji}$ 的对称性。

  \skiplines
  
    我们还经常需要把联络 $\Gamma_{kij}$ 的第一个指标升上去,变成
  {\blue
  $$\Gamma^k_{\ ij} \equiv g^{kl}\Gamma_{lij} = \vecx^{,k}\cdot\vecx_{,i,j}$$}

\end{frame}


\begin{frame}
  \frametitle{联络可以用度规导出}
  度规的偏导数可以用联络写出来:
  $$g_{ij,k} = \left(\vecx_{,i}\cdot\vecx_{,j}\right)_{,k} = \vecx_{,i,k}\cdot\vecx_{,j} + \vecx_{,i}\cdot\vecx_{,j, k}  = \Gamma_{jik}+\Gamma_{ijk}.$$
  同理
  $$ g_{ki,j} = \Gamma_{ijk}+\Gamma_{kij}$$
  $$ g_{kj,i} = \Gamma_{jik}+\Gamma_{kij}$$
  把后两式相加并减去第一式,即可得到:
  \tbox{$$\Gamma_{kij} = \frac{1}{2}\left(g_{ki,j}+g_{kj,i}-g_{ij,k}\right).$$}
  也就是说,联络可以由第一基本形式得到。
\end{frame}


\begin{frame}
  \frametitle{升级武器}
  我们再把秘密武器进行一下升级:任意两个三维欧式空间的矢量 $\vecy$, $\vecz$。可以把它们各自按秘密武器展开后,再求两者的内积
  \bea
  \vecy\cdot\vecz &=& \left[\left(\vecx_{,i}\cdot\vecy\right)\vecx^{,i}+(\vecn\cdot\vecy)\vecn\right]\cdot\left[\left(\vecx_{,j}\cdot\vecz\right)\vecx^{,j}+(\vecn\cdot\vecz)\vecn\right] \newl
  &=& \left(\vecx_{,i}\cdot\vecy\right)\left(\vecx_{,j}\cdot\vecz\right)\vecx^{,i}\cdot\vecx^{,j}  + (\vecn\cdot\vecy)(\vecn\cdot\vecz) \newl
  &=& \left(\vecx_{,i}\cdot\vecy\right)\left(\vecx_{,j}\cdot\vecz\right)g^{ij} + (\vecn\cdot\vecy)(\vecn\cdot\vecz) \newl
  &=& \left(\vecx_{,i}\cdot\vecy\right)\left(\vecx^{,i}\cdot\vecz\right) + (\vecn\cdot\vecy)(\vecn\cdot\vecz)    
  \eea
  我们得到了一个计算 $(\vecn\cdot y)(\vecn\cdot z)$ 的有效方法:
  {\blue
     $$ (\vecn\cdot\vecy)(\vecn\cdot\vecz)   =  \vecy\cdot\vecz - \left(\vecx_{,i}\cdot\vecy\right)\left(\vecx^{,i}\cdot\vecz\right).$$
  }

\end{frame}



\begin{frame}
  \frametitle{计算高斯曲率 $K=\frac{\det(\beta)}{\det(g)}$}
  利用升级的秘密武器,立刻有
 {\small
  \bea
  && \det(\beta) \newl
  &=& \beta_{11}\beta_{22} - (\beta_{12})^2 \newl
  &=& \left(\vecn\cdot\vecx_{,1,1}\right)\left(\vecn\cdot\vecx_{,2,2}\right) - \left(\vecn\cdot\vecx_{,1,2}\right)^2 \newl
  &=& \vecx_{,1,1}\cdot\vecx_{,2,2} - \left(\vecx_{,i}\cdot\vecx_{,1,1}\right)\left(\vecx^{,i}\cdot\vecx_{,2,2}\right) - \left(\vecx_{,1,2}\right)^2 +  \left(\vecx_{,i}\cdot\vecx_{,1,2}\right)\left(\vecx^{,i}\cdot\vecx_{,1,2}\right) \newl
  &=& \vecx_{,1,1}\cdot\vecx_{,2,2}  - \Gamma_{i11}\Gamma^i_{\ 22} -  \left(\vecx_{,1,2}\right)^2 + \Gamma_{i12}\Gamma^i_{\ 12} \newl 
  &=& (\vecx_{,1,1}\cdot\vecx_{,2})_{,2}-  \left(\vecx_{,1,2}\cdot\vecx_{,2}\right)_{,1} + \Gamma_{i12}\Gamma^i_{\ 12} - \Gamma_{i11}\Gamma^i_{\ 22} \newl
  &=& \left(\Gamma_{211}\right)_{,2} -  \left(\Gamma_{212}\right)_{,1} + \Gamma_{i12}\Gamma^i_{\ 12} - \Gamma_{i11}\Gamma^i_{\ 22} \newl    
 \eea
 }
 联络可以用度规和度规的偏导数写出来,所以任务完成。
\end{frame}


\begin{frame}
  \frametitle{sympy又来了}
  从度规到联络和从联络到高斯曲率的计算都比较繁琐。我们可以写一个 sympy 脚本代劳,请参考:

  \url{http://zhiqihuang.top/gr/codes/connection.py}
\end{frame}


\ech
\end{document}
