\documentclass[CJK,13pt]{beamer}
\input{macros.tex}
\def\courseurl{http://zhiqihuang.top/gr}

\def\tpage#1#2{
\title{GR \S{#1}  #2}
  \author{Zhiqi Huang}

\begin{frame}
\begin{center}
{\bf \Huge G}eneral {\bf \Huge R}elativity

{\vskip 0.1in}



{\Large \S #1 #2}

{\vskip 0.2in}

{Lecturer: 黄志琦}

\vskip 0.2in

\courseurl

\end{center}
\end{frame}
}


  \date{}
  \begin{document}
  \bch
  \tpage{24}{Cosmology - FRW metric}

  \begin{frame}
    \frametitle{Mach的思想实验}

    \addfig{1.5}{mach_experiment.jpg}

    当水桶相对宇宙的恒星/星系背景没有旋转时,离心力就消失了,这是惊人的巧合吗?
  \end{frame}


  \begin{frame}
    \frametitle{标准宇宙学的基本图像}
    \bmini{0.42}
    宇宙由一个非常均匀且各向同性的初始状态开始,相距很远的“测地线粒子”间的距离用标准尺子——普朗克长度来度量,不断地增大(即所谓的宇宙大爆炸)。不过,邻近的粒子间由于引力的不稳定性,逐渐结团。
    \emini
    \bmini{0.53}
    \addfig{2}{large_scale_structure.jpg}
    \emini

    \skiplines

    虽然引力导致物质在小尺度上结团,在亿光年为单位的大尺度上看,宇宙仍然均匀和各向同性。    

  \end{frame}

\begin{frame}
  \frametitle{宇宙学基本原理(cosmological principle)}
  从天圆地方,到地心说,再到日心说,再到发现河外星系……一次又一次的沉重打击之后,人类基本上已经放弃了把自己置于宇宙中心的想法。

  \addfig{2}{universe.jpg}
  

  
  \skipline
  
  宇宙学基本原理:\sout{既然我们做不了宇宙中心,那谁也别想做。}{\blue 宇宙中没有任何特殊观测者,大尺度上看,宇宙均匀且各向同性。}
  {\scriptsize (其实叫基本假设更合理)}
\end{frame}
  

\begin{frame}
  \frametitle{FRW 度规}
  满足均匀且各向同性的只有 Friedmann–Lema\^itre–Robertson–Walker 度规,有时候也简称 Friedmann–Robertson–Walker 或 FLRW 或 FRW 度规:

  {\scriptsize Lema\^itre 的失败可能在于英文键盘打不出他的名字……}
  
  $$ ds^2 = dt^2 - a^2(t)\left(\frac{dr^2}{1-kr^2}+r^2d\theta^2+r^2\sin^2\theta d\phi^2\right)$$
  这里的 $a(t)$ 是任意一个 $t$ 的光滑函数。$k$ 是代表着三维空间的曲率的常数。

  \skipline
  
  为了书写方便,在这一讲里我们作一些约定:
  \bitem
\item{用符号上一点表示对宇宙学时间$t$的导数:如 $\dot{a}$ 表示 $\frac{da}{dt}$。}
\item{用下标``0''表示一个参量的“目前”值。例如$t_0$ 是“目前”的宇宙学时间。}
  \eitem
\end{frame}

\begin{frame}
  \frametitle{共动坐标}
  由于 $\Gamma^r_{\ tt}=\Gamma^\theta_{\ tt} =\Gamma^{\phi}_{\ tt}=0$,如果一开始测试粒子的 $p^r=p^\theta=p^\phi = 0$,则根据测地线方程
  $$\frac{d p^\alpha}{dt} + \Gamma^{\alpha}_{\mu\nu}\frac{p^\mu p^\nu}{p^0} = 0$$
  这里的 $\alpha=r, \theta,\phi$,无论 $\mu,\nu$ 同时为 $t$ (这样 $\Gamma^{\alpha}_{\mu\nu}$ 为零),还是 $\mu,\nu$ 至少有一个不是 $t$(这样 $p^\mu p^\nu$ 为零),都会使 $\frac{d p^\alpha}{dt}$ 为零。

      \addfig{2}{chuangmoli.jpg}
      
  也就是说 ,测地线粒子在坐标 $(r,\theta,\phi)$ 可以一直保持“静止不动”,我们把 $r,\theta,\phi$ 叫做“共动坐标”。  
\end{frame}


\begin{frame}
  \frametitle{尺度因子 $a$ 的归一化是任意的} 
  FRW度规里的 $a(t)$ 称为{\blue 宇宙学尺度因子(scale factor)}。

  \skipline
  
  $a(t)r$ 具有物理长度的意义。如果把 $a(t)$ 多乘一个常数,同时把 $r$ 除以这个常数,并把 $k$ 乘以这个常数的平方,则不改变任何物理内容。我们经常把 $a_0$ (今天的 $a$)取成 $1$。
  
\end{frame}


\begin{frame}
  \frametitle{空间曲率 $k$}
  用固定的宇宙学时间 $t$ 在FRW时空切出一个三维空间,并把该时刻的 $a$ 归一化为 $1$,则三维空间的度规为:
  $$\left.ds^2\right\vert_{3D} = \frac{dr^2}{1-kr^2}+r^2d\theta^2+r^2\sin^2\theta d\phi^2.$$
  这时可以计算出 Ricci标量为 $6k$。

  \skiplines
  
  我们把 $k>0$ (正曲率空间)的宇宙称为闭合宇宙(closed universe),$k<0$ 的宇宙叫做开放宇宙(open universe),把 $k=0$ 的宇宙叫做平坦宇宙(flat universe)。

  {\scriptsize 注意“平坦”只是指空间平坦,整个四维时空还是弯曲的。}
  
\end{frame}

\begin{frame}
\frametitle{平坦宇宙和观测数据(特别是宇宙微波背景辐射数据)符合得还不错}
  \addfig{3}{spatial_curvature.jpg}  
\end{frame}

\begin{frame}
  \frametitle{哈勃参数}
  哈勃参数的定义为:
  \be
  H = \frac{\dot a}{a}
  \ee
  在标准宇宙学里,它恒大于零(即宇宙一直在膨胀)。

  \skipline

  $H_0$,也就是在 $t=t_0$ 时刻的 $H$,叫做{\blue 哈勃常数}。它大致在 $70\mathrm{km/s/Mpc}$ 左右。

  {\scriptsize 这里的长度单位 $\mathrm{Mpc}$ 是百万秒差距的意思,一个秒差距大约是 $3.26$ 光年。}

\end{frame}


\begin{frame}
  \frametitle{Hubble tension}
  \addfig{3.5}{H0tension.png}
\end{frame}

\begin{frame}
  \frametitle{思考题}
  \addfig{0.5}{think5.jpg}
  
  假设一个共动观测者在 $(t_1, r_1, \theta_1,\phi_1)$ 发射了一个TA看来频率为 $\nu$ 的光子。在 $(t_2, r_2,\theta_2, \phi_2)$ 处一个共动观测者接收到光子,接收者看到光子的频率是多少?
\end{frame}


\begin{frame}
  \frametitle{宇宙学红移}
  共动观测者的四维速度为 $u_\mu = (1, 0, 0, 0)$,因此看到的光子频率直接和光子在FRW坐标系的 $p^t$ 成正比。光子的测地线方程(见附录的联络表达式)为:
  $$ \frac{dp^t}{dt} + \frac{a\dot a}{p^t} \left[\frac{(p^r)^2}{1-kr^2} + r^2\left(p^\theta\right)^2 + r^2\sin^2\theta \left(p^\phi\right)^2\right] = 0.$$
    又光子走的是零测地线,所以:
    $$\left(p^t\right)^2 - a^2\left[\frac{(p^r)^2}{1-kr^2} + r^2\left(p^\theta\right)^2 + r^2\sin^2\theta \left(p^\phi\right)^2 \right] = 0.$$
    于是就有
    $$\frac{d p^t}{dt}  + \frac{\dot a}{a } p^t = 0$$
    它的解显然是 $p^t \propto \frac{1}{a} $。也就是说,光子的波长和 $a$ 成正比。
\end{frame}


\begin{frame}
  \frametitle{宇宙学红移}
  宇宙学红移就定义为波长的相对变化量
  $$z \equiv \frac{\lambda_{\rm obs}-\lambda_{\rm emit}}{\lambda_{\rm emit}}$$
  通常观测者指的都是目前地球上的观测者。所以在宇宙学时间 $t$ 发射出的光子的宇宙学红移为:
  $$ z = \frac{a_0-a}{a}, $$
  这里的 $a=a(t)$ 是发射点的尺度因子。 容易看出
  $$  \frac{a}{a_0} = \frac{1}{1+z}.$$
\end{frame}


\begin{frame}
  \frametitle{观测量的红移}
  在天文上,经常看到“这个星系的红移是$z$”这种说法。它的意思是:我们收集到的来自星系的光子,是星系在
  $$ \frac{a(t)}{a_0} = \frac{1}{1+z}$$
  对应的宇宙学时间 $t$ 时刻发出的(假设星系足够远,可以近似看成一个共动测试粒子)。

  \skipline

  显然,$z$ 越大,代表看到的星系越古老,离我们的距离也越远。
\end{frame}


\begin{frame}
  \frametitle{Friedmann Equations}
  如果宇宙的物质可以看成均匀的理想流体,密度为 $\rho(t)$, 压强为 $p(t)$,则爱因斯坦方程可以简化为:

  \bea
  \frac{k + \dot a ^2}{a^2} &=& \frac{8\pi G}{3}\rho \newl
  \frac{\ddot a}{a} &=& -\frac{4\pi G}{3}\left(\rho + 3p\right)  
  \eea
  这两个方程分别称为第一个和第二个 Friedmann 方程
  
\end{frame}


\begin{frame}
  \frametitle{附录1:FRW度规的联络}
\bmini{0.48}
\bea
\Gamma^t_{\ rr} &=& \frac{a\dot a}{1-kr^2} \newl
\Gamma^t_{\ \theta \theta } &=& r^2a\dot a \newl
\Gamma^t_{\ \phi \phi } &=& r^2a\dot a \sin^2\theta \newl
\Gamma^r_{\ r t} &=& \frac{\dot a}{a} \newl 
\Gamma^r_{\ rr} &=&  \frac{k r}{1-k r^2}  \newl
\Gamma^r_{\ \theta \theta } &=& -r(1-kr^2) \newl
\Gamma^r_{\ \phi \phi } &=& -r \left(1-k r^2 \right) \sin^2\theta
\eea
\emini
\bmini{0.48}
\bea
\Gamma^\theta _{\ \theta  t} &=& \frac{\dot a}{a} \newl
\Gamma^\theta _{\ \theta r} &=& \frac{1}{r} \newl
\Gamma^\theta _{\ \phi \phi } &=& - \sin\theta \cos\theta \newl
\Gamma^\phi _{\ \phi  t} &=& \frac{\dot a}{a} \newl
\Gamma^\phi _{\ \phi r} &=& \frac{1}{r} \newl
\Gamma^\phi _{\ \phi \theta } &=& \cot\theta
\eea
\emini
\end{frame}


\begin{frame}
  \frametitle{附录2:FRW度规的Ricci张量和Einstein 张量}
  Ricci tensor:
  \bea
R^{ t}_{\ t}&=& - \frac{3 \ddot a}{a} \newl
R^{r}_{\ r}&=&R^{\theta }_{\ \theta } =R^{\phi }_{\ \phi }= - \frac{2 k + a \ddot a + 2 \dot a^2}{a^2} 
\eea
Ricci scalar:
\be
R= - \frac{6\left(k + a \ddot a + \dot a^2\right)}{a^2} 
\ee
Einstein tensor:
\bea
G^{ t}_{\ t}&=& \frac{3\left(k + \dot a ^2\right)}{a^2}  \nonumber \\
G^{r}_{\ r}&=& G^{\theta }_{\ \theta }= G^{\phi }_{\ \phi } =\frac{k + 2 a \ddot a + \dot a ^2}{a^2}  
\eea

\end{frame}


    \ech
\end{document}




  
