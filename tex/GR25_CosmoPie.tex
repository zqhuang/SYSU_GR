\documentclass[CJK,13pt]{beamer}
\input{macros.tex}
\def\courseurl{http://zhiqihuang.top/gr}

\def\tpage#1#2{
\title{GR \S{#1}  #2}
  \author{Zhiqi Huang}

\begin{frame}
\begin{center}
{\bf \Huge G}eneral {\bf \Huge R}elativity

{\vskip 0.1in}



{\Large \S #1 #2}

{\vskip 0.2in}

{Lecturer: 黄志琦}

\vskip 0.2in

\courseurl

\end{center}
\end{frame}
}


  \date{}
  \begin{document}
  \bch
  \tpage{25}{Cosmological Pie}

\begin{frame}
  \frametitle{缺一个方程}
  Friedmann方程:
  \bea
  \frac{k + \dot a ^2}{a^2} &=& \frac{8\pi G}{3}\rho \newl
  \frac{\ddot a}{a} &=& -\frac{4\pi G}{3}\left(\rho + 3p\right)  
  \eea
  有三个未知函数 $a(t), \rho(t), p(t)$,因此还需要一个状态方程描述 $p$ 和 $\rho$ 的关系。
\end{frame}

\begin{frame}
  \frametitle{标准宇宙学认为的{\bf 目前}宇宙组成}
  \addfig{2.5}{pie.pdf}
\end{frame}


\begin{frame}
  \frametitle{压力\sout{山}$\frac{1}{3}$大的辐射能量(radiation)}
  以光子为例:

  \skipline
  
  “共动体积”内的光子数守恒,所以(单位物理体积内的)光子数密度和 $a^3$ 成反比。同时宇宙学红移使得单个光子的能量和 $a$ 成反比。最终就有
  $$\rho_r \propto a^{-4}$$
  另外极端相对论性的气体的压强
  $$ p_r = \frac{\rho_r}{3} $$
  
  在不特别关注中微子的情况下,经常把中微子也当成辐射形式的能量。
  
  {\scriptsize 在低红移$z\lesssim 1$处这可能不是很好的近似,但是低红移的中微子仅占总能量的$0.1\%$,问题不是很大。}
\end{frame}


\begin{frame}
  \frametitle{毫无压力的物质粒子(matter)}
  如果是静止质量为主的物质粒子(matter),其数密度同样和 $a^3$ 成反比,但是就可以不用考虑宇宙学红移。最终就有
  $$\rho_m \propto a^{-3}.$$
  在大尺度上忽略粒子运动,即近似有
  $$p_m\approx 0.$$
  这里的物质通常指粒子物理标准模型包含的普通物质和尚不清楚是啥的冷暗物质。
\end{frame}


\begin{frame}
  \frametitle{压力为负的宇宙学常数($\Lambda$)}
  宇宙学常数$\Lambda$是暗能量的标准解释。它的能量密度是常数:
  $$\rho_{\Lambda} = \mathrm{const.}$$
  它的压强是负的
  $$ p_{\Lambda} = -\rho_{\Lambda}.$$
\end{frame}


\begin{frame}
  \frametitle{思考题}
  \addfig{0.5}{think1.jpg}
  
  如果某个理想流体的压强和能量密度的关系为 $p=w\rho$,这里的 $w$ 为常量。对固定共动半径为 $r\ll \frac{1}{\sqrt{k}}$的球应用能量守恒:
  $$ d\left(\rho \frac{4\pi a^3r^3}{3}\right) = -p(4\pi a^2r^2) d(ar) $$
  证明
  $$ \rho \propto a^{-3(1+w)} $$
  对辐射形式能量($w=1/3$),冷物质($w=0$),和宇宙学常数($w=-1$)分别验证上式。
\end{frame}


\begin{frame}
  \frametitle{更严格的证明}
  对理想流体
  $$T^{\mu}_{\ \nu} = \diag(\rho, -p, -p, -p) $$
  (参考第14讲)
  \bea
  T^{\mu}_{\ 0;\mu} &=& T^\mu_{\ 0,\mu} + \Gamma^\mu_{\ \alpha\mu}T^\alpha_{\ 0} - \Gamma^\alpha_{\ 0\mu} T^\mu_{\ \alpha} \newl
  &=& \dot\rho + 3H(\rho + p) \newl
  \eea
      (参考前一讲的FRW度规的联络表达式)
      
      于是 $T^{\mu}_{\ 0;\mu}=0$ 可以转化为
      $$\dot\rho + 3H(\rho + p) = 0$$
      请证明上式和前面思考题中的“能量守恒”式是等价的。
\end{frame}


\begin{frame}
  \frametitle{无量纲能量密度参数 $\Omega_x$}
  对任一成分 $x$,记{\blue
    $$ \Omega_x \equiv \frac{8\pi G\rho_x}{3H_0^2} $$}
  并特别地对空间曲率 $k$ 规定{\blue
    $$\Omega_k \equiv - \frac{k}{a_0^2H_0^2} $$}
  由 $t_0$ 时刻的 Friedmann 方程可以得到{\blue
  $$ \sum_{x}\Omega_x = 1$$}
  这里的 $x$ 遍历所有成分以及空间曲率 $k$。
\end{frame}


\begin{frame}
  \frametitle{数量级}
  现在宇宙学的理论及观测给出
  $$\Omega_m\approx 0.3, \Omega_\Lambda\approx 0.7, \Omega_r\approx 9\times 10^{-5}, \Omega_k\approx 0$$
  当然,在研究“非标准”的可能性的时候,这些都是允许被改变的。
\end{frame}

\begin{frame}
  \frametitle{任意红移处的哈勃参数}
  利用各种成分和 $a$ 之间的幂率关系,以及第一个 Friedmann 方程,容易得到任意红移 $z$ 处的哈勃参数:{\blue
    $$H(z) = H_0\sqrt{\Omega_\Lambda + \Omega_k(1+z)^2+\Omega_m(1+z)^3+\Omega_r(1+z)^4}$$}
  如果按通常约定 $a_0=1$,哈勃参数也可以写成尺度因子的函数{\blue
    $$H(a) = H_0\sqrt{\Omega_\Lambda + \Omega_ka^{-2}+\Omega_ma^{-3}+\Omega_ra^{-4}}$$}  
\end{frame}


\thinka{对 $\Omega_m=1, \Omega_\Lambda=0, \Omega_k=\Omega_r=0$,计算宇宙年龄。}

\thinkb{对 $\Omega_m=0.3, \Omega_\Lambda=0.7, \Omega_k=\Omega_r=0$,计算宇宙年龄。}


\thinkf{对 $\Omega_m=0.3, \Omega_\Lambda=0.7, \Omega_k=\Omega_r=0$,计算目前宇宙的“减速因子” $$q\equiv -\frac{a\ddot a}{\dot a^2}$$的值。}

    \ech
\end{document}




  
