\documentclass[CJK,13pt]{beamer}
\input{macros.tex}
\def\courseurl{http://zhiqihuang.top/gr}

\def\tpage#1#2{
\title{GR \S{#1}  #2}
  \author{Zhiqi Huang}

\begin{frame}
\begin{center}
{\bf \Huge G}eneral {\bf \Huge R}elativity

{\vskip 0.1in}



{\Large \S #1 #2}

{\vskip 0.2in}

{Lecturer: 黄志琦}

\vskip 0.2in

\courseurl

\end{center}
\end{frame}
}


  \date{}
  \begin{document}
  \bch
\tpage{3}{Embedded Surface}


\begin{frame}
  \frametitle{正则曲面(regular surface)}
  设$n$维欧氏空间有曲面 $\vecx(u, v) = \left(x_1(u, v), x_2(u, v), \ldots, x_n(u, v)\right)$。

  \skipline
  
  本讲中我们将始终假设这些函数至少三次可导,并且矢量 $\frac{\partial\vecx}{\partial u}$ 和 $\frac{\partial\vecx}{\partial v}$ 始终线性无关(即两者均非零且方向不同)。这样的曲面称为正则曲面。

  \skiplines

  {\scriptsize 严格来讲,上面定义的是“正则曲面片”。一个正则曲面可以由多块正则曲面片光滑地拼接起来,但整体未必可用统一的满足正则曲面片要求的 $\vecx(u,v)$ 参数形式来表述。在大多数情形下,为了表述简便,我们忽略这些名词的细节差异。}
\end{frame}

\begin{frame}
  \frametitle{曲面的第一基本形式(1st Fundamental Form)}
  当从曲面上一个点 $\vecx(u,v)$ 移动到一个邻近的点 $\vecx(u+du, v+dv)$,则有微分表达式:
  
  \bmini{0.5}
  $$ d\vecx = \frac{\partial\vecx}{\partial u} du + \frac{\partial\vecx}{\partial v} dv. $$
  \emini
  \bmini{0.45}
  \lfig{2}{firstform.png}
  \emini
  
  由此得到曲面的第一基本形式:
{\blue  $$d\vecx^2 = \left(\frac{\partial\vecx}{\partial u}\right)^2 du^2 +2\left(\frac{\partial\vecx}{\partial u}\cdot\frac{\partial\vecx}{\partial v}\right)du dv+ \left(\frac{\partial\vecx}{\partial v}\right)^2 dv^2.$$}
  线性代数的常识告诉我们:这个表达式是 $du, dv$ 的正定二次型,而且在变量替换 $u,v\rightarrow \tilde{u}, \tilde{v}$ 下是不变量。
\end{frame}


\begin{frame}
  \frametitle{美好回忆}
  为了唤起你对线性代数的美好回忆,我特地把第一基本形式写成了这样的形式:

  $$ d\vecx^2  =
  \begin{pmatrix}
    du &    dv
  \end{pmatrix}
  \begin{pmatrix}
    g_{uu} &   g_{uv} \\
    g_{vu} & g_{vv}
  \end{pmatrix}
  \begin{pmatrix}
    du \\
    dv
  \end{pmatrix}  
  $$
  这里的{\blue 度规矩阵}
  $$ \begin{pmatrix}
    g_{uu} &   g_{uv} \\
    g_{vu} & g_{vv}
  \end{pmatrix}
  $$
  定义为
  $$g_{uu}= \left(\frac{\partial\vecx}{\partial u}\right)^2,\ g_{uv} = g_{vu} = \frac{\partial\vecx}{\partial u}\cdot\frac{\partial\vecx}{\partial v},\ g_{vv} = \left(\frac{\partial\vecx}{\partial v}\right)^2.$$
  
\end{frame}



\begin{frame}
  \frametitle{来来来再看会儿图}
  请仔细凝视这个图
  
  \addfig{2}{firstform.png}
  
  你还能写出曲面上的面积分:
  $$ S = \int \sqrt{\det g} du dv $$
  这里的 $\det g$表示度规矩阵的行列式(请检验它是图中平行四边形的面积的平方)
\end{frame}



\begin{frame}
  \frametitle{二阶微分}
  曲面 $\vecx(u,v)$ 在某点 $(u_0, v_0)$ 的切平面可以写成
  $$ \vecy(u, v) = \vecx(u_0,v_0) + \frac{\partial \vecx(u_0,v_0)}{\partial u_0} (u-u_0)  + \frac{\partial \vecx(u_0,v_0)}{\partial v_0} (v-v_0).$$

  \skiplines
  
  曲面在切点附近偏离切平面的量近似为二阶微分小量 $\frac{1}{2}d^2\vecx$,这里的
  $$ d^2\vecx = \frac{\partial^2 \vecx}{\partial u^2} du^2 + 2\frac{\partial^2 \vecx}{\partial u\partial v} du dv + \frac{\partial^2 \vecx}{\partial v^2} dv^2.$$
  很可惜,二阶微分是非线性操作,所以线性代数大法失效——一般来说,上面的表达式在变量替换下{\bf 不是}不变量。
\end{frame}


\begin{frame}
  \frametitle{曲面的第二基本形式(2nd Fundamental Form)}
  现在开始我们把范围限定在三维空间里的正则曲面。

  这时曲面在一点具有唯一的(单位)法向量:
{\blue  $$\vecn = \frac{\frac{\partial\vecx}{\partial u}\times \frac{\partial\vecx}{\partial v}}{\lvert\frac{\partial\vecx}{\partial u}\times \frac{\partial\vecx}{\partial v}\rvert}.$$  }
  也许有些出乎你的意料:二阶微分 $d^2\vecx$ 在曲面的单位法向量上的投影 $\vecn\cdot d^2\vecx $ 是个不变量。这是因为:
  $$  \vecn \cdot d^2\vecx = d( \vecn \cdot d\vecx) - d\vecn \cdot d\vecx = - d\vecn \cdot d\vecx.$$
  我们用到了 $\vecn \cdot d\vecx = 0$,即 $\vecn$ 和切平面垂直这个事实。
  
  因为二次型 $-d\vecn\cdot d\vecx$ 在坐标变换下是不变量,所以 $\vecn\cdot d^2\vecx$ 也是不变量——我们称之为曲面的第二基本形式。
\end{frame}

\begin{frame}
  \frametitle{平面判据}
  具体写出第二基本形式就是:
{\blue $$\vecn\cdot d^2\vecx =\left(\vecn\cdot \frac{\partial^2 \vecx}{\partial u^2}\right) du^2 + 2\left(\vecn\cdot\frac{\partial^2 \vecx}{\partial u\partial v}\right) du dv + \left(\vecn\cdot\frac{\partial^2 \vecx}{\partial v^2}\right) dv^2.$$}
  
  根据第二类基本形式的几何意义(曲面偏离切平面的法向分量),显然有
  
  {\blue 在三维欧式空间里,曲面是平面的充要条件是:第二基本形式恒为零。}
\end{frame}



\begin{frame}
  \frametitle{曲面论的基本定理}
  在三维欧式空间中,曲面的第一基本形式描述了曲面在切向的度量(类比曲线的弧长参数),第二基本形式完整地描述了曲面是如何偏离切平面的(类比曲线的曲率和挠率)。根据这些直观图像,可以得出曲面论的基本定理:

  \skipline
  
  {\blue
    如果在三维欧式空间中两个曲面有完全相同的第一基本形式和第二基本形式,则它们一定可以通过一个刚体运动完全重合起来。
    }
\end{frame}


\begin{frame}
  使用第二基本形式可以轻松判断曲面的弯曲程度,但是如果只有第一基本形式,无法获得第二基本形式呢?这要从一个二维爬虫的故事讲起——
\end{frame}


\begin{frame}
  \frametitle{二维爬虫的故事}
  假设有活在曲面上的爬虫,它们使用 $(u, v)$ 来记录世界的位置。它们并不知道高维空间里垂直于曲面的法向量的存在,也就是说,它们根本就没有曲面的第二基本形式的概念。

  \addfig{2}{bugonsurface.jpg}

  它们是不是可以测量曲面的第一基本形式呢?注意它们的行动都限制在曲面上,所以像“直接测量三维坐标来定距离”之类的操作肯定不行。

\end{frame}  


\begin{frame}
  \frametitle{二维爬虫的故事}
  如果它们有把不依赖于曲面性质的尺子,问题就容易解决了。
  
  \skipline
  
  这把标准尺子可以用量子力学的普朗克常量 $\hbar$,万有引力常数 $G$,以及真空中的光速 $c$ 构造出来: 普朗克长度 $\ell_p = \sqrt{\hbar G/c^3}$。也就是说,早在爬虫们对自己所在世界的几何发生兴趣之前,它们已经发现了量子力学、引力平方反比律、以及光速不变原理。

  \skipline
  
  它们用$\ell_p$作为长度单位对自己所在的空间进行几何测量。并发现自己的世界相邻两点间距离具有
  $$ ds^2 = g_{uu}du^2 + 2g_{uv}du dv + g_{vv}dv^2 $$
  的形式。  
\end{frame}



\begin{frame}
  \frametitle{二维爬虫的故事}
  在爬虫们进行这些测量之前,它们内心独白是:嗯,应该只有最简单最优美的平面才配得上这个拥有智能爬虫的伟大世界。

  \skipline

  在平面上,勾股定理处处成立:如果 $u, v$ 是两个直角坐标,度规矩阵应该是单位矩阵。

  \skipline
  
  但测量结果让爬虫们大吃一惊,它们发现 $g_{uu}, g_{uv}, g_{vv}$ 这些量都是 $u, v$ 的比较复杂的函数。

  \addfig{1.3}{bengkui.jpg}
\end{frame}


\begin{frame}
  \frametitle{二维爬虫的故事}  
  \addfig{1.5}{liangjiaoqi.png}

  \bitem
\item{爬虫们首先怀疑自己选取的 $u, v$ 是不是偏离了直角坐标。它们试图重新选取坐标 $u, v \rightarrow \tilde{u},\tilde{v}$ 使得勾股定理处处成立。但是它们无论怎样尝试都没有获得成功。}
  \eitem

  \skiplines
  
  \lfig{0.5}{think2.jpg}有没有一个方法,可以帮助它们直接用测到的第一基本形式判定它们生活的世界不是一个平面?
\end{frame}

\begin{frame}
  \frametitle{二维爬虫的故事}
  \bitem
\item{然后它们怀疑是不是量子力学,万有引力定律,或者光速不变原理这些物理定律出现了问题。}
  \eitem

  \addfig{1.1}{minke.jpg}

但是它们很快发现自己和民科一样逻辑混乱了……
\end{frame}


\begin{frame}
  把二维曲面换成四维时空,把爬虫换成人类。

  \skiplines
  
  {\blue 仅使用第一基本形式来研究时空的弯曲程度,就是我们在《广义相对论》里要干的事情。}

\end{frame}


\ech
\end{document}
