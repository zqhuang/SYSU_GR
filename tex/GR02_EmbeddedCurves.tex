\documentclass[CJK,13pt]{beamer}
\input{macros.tex}
\def\courseurl{http://zhiqihuang.top/gr}

\def\tpage#1#2{
\title{GR \S{#1}  #2}
  \author{Zhiqi Huang}

\begin{frame}
\begin{center}
{\bf \Huge G}eneral {\bf \Huge R}elativity

{\vskip 0.1in}



{\Large \S #1 #2}

{\vskip 0.2in}

{Lecturer: 黄志琦}

\vskip 0.2in

\courseurl

\end{center}
\end{frame}
}


  \date{}
  \begin{document}
  \bch
\tpage{2}{Embedded Curves}


\begin{frame}
  \frametitle{欧式空间中的正则曲线}
  设$n$维欧氏空间有条曲线 $\vecx(t) = \left(x_1(t), x_2(t), \ldots, x_n(t)\right)$。

  \skipline

  它的切向量可以定义为 $\frac{d\vecx}{dt} = \left(\frac{dx_1}{dt}, \frac{dx_2}{dt}, \ldots, \frac{dx_n}{dt}\right)$,这也是一个向量函数。

  \skipline
  
  本讲中我们将一直假设曲线的各个分量函数至少三次可导,且切向量处处非零: $\left\vert\frac{d\vecx}{dt}\right\vert\ne 0$。这样的曲线称之为“{\blue 正则曲线}”。

\end{frame}



\begin{frame}
  \frametitle{曲线的弧长参数和单位切向量$\vecT$}
  如果参数从某个允许的确定值 $a$ (例如,如果允许的话,可取 $a=0$) 变化到 $t$, 对应的一段弧长为
  $$ s = \int_a^t \left\vert\frac{d\vecx}{d\tau}\right\vert d\tau. $$
  我们可以选取 $s$ 而不是 $t$ 作为参量来描述曲线。
  
  {\scriptsize 注1:正则性保证了 $t$ 到 $s$ 的映射是单调映射。}

  \skipline

  {\scriptsize 注2:一般我们假定曲线的参数形式默认给了曲线一个方向,即 $t$ 的正向给出了曲线的方向。选定方向的曲线的弧长参数 $s$ 几乎就是确定的了,至多相差一个常数。我们后面讨论的 Frenet 标架等,都依赖于曲线方向已经给定这个假定。否则曲线的弧长参数和 Frenet 标架会有个符号不确定性。}

  \skipline
  
  选择了弧长参数之后,显然地, $\vecT \equiv \frac{d\vecx}{ds}$ 是{\blue 单位切向量}。  
\end{frame}


\begin{frame}
  \frametitle{曲率(curvature)}

  曲线的一小段总是能用圆弧来近似。圆弧半径越大,曲线的一小段就看起来越直。反之,圆弧半径越小,曲线的一小段就看起来很弯曲。所以曲线在一点上的{\blue “曲率” $\kappa$ 就定义为这点附近的近似圆弧的半径的倒数}。
  
  \addfig{2}{curvature.jpg}

\end{frame}



\begin{frame}
  \frametitle{曲线的曲率公式}

  假设有曲线 $\vecx(s)$, $s$ 是弧长参数。在曲线上的给定点附近, 单位切向量 $\frac{d\vecx}{ds}$ 的变化的大小应该等于它的方向变化的大小(用近似圆弧来计算就是 $\frac{ds}{R}$)。
  
  $$ \left\vert d \frac{d\vecx}{ds}\right\vert = \frac{ds}{R} $$


  \addfig{2}{curvature_explain.jpg}

\end{frame}

\begin{frame}
  \frametitle{曲线的曲率公式}
  由此得到曲率 $\frac{1}{R}$ 的值
  {\blue
  $$\kappa =  \left\vert\frac{d^2\vecx}{ds^2}\right\vert. $$}

  这个公式如此简洁美妙,完全得益于我们选取了弧长作为曲线的参数。

  \skipline
  
  如果我们把取绝对值的操作去掉,则得到的是一个“曲率向量”:
  $$\frac{d^2\vecx}{ds^2}. $$
  它的大小为曲率,方向从所讨论的点指向近似圆弧的圆心。  


\end{frame}


\begin{frame}
  \frametitle{密切平面, 主法向量,次法向量}
  假设曲线上某点的曲率向量 $\frac{d^2\vecx}{ds^2}$ 非零 (显然它和该点单位切向量 $\vecT = \frac{d\vecx}{ds}$ 垂直), 
  我们称归一化的曲率向量,也就是:
  $$ \vecN \equiv \frac{1}{\kappa} \frac{d^2\vecx}{ds^2}$$
  为曲线在该点的{\blue 主法向量}(normal vector)。


  单位切向量 $\vecT$ 和主法向量 $\vecN$ 一起确定了近似圆弧所在的“{\blue 密切平面}”; 和密切平面垂直的 $n-2$ 个单位法向量则称为{\blue 次法向量} (binormal vectors)。
  
\end{frame}

\begin{frame}
  \frametitle{三维空间曲线的Frenet标架}


  从现在开始我们专门讨论嵌入在三维空间的曲线。这时单位切向量$\vecT$,主法向量$\vecN$,次法向量$\vecB$一起构成了一个局域的直角坐标架——它有个很炫酷的名字,叫“{\blue Frenet 标架}”。  

  \addfig{2.7}{FrenetFrame.png}
  

\end{frame}


\begin{frame}
  \frametitle{嵌在三维空间的曲线的Frenet标架总结}  
  切向量(tangent vector)
  $$\vecT(s) = \frac{d\vecx}{ds};$$
  主法向量(normal vector)
  $$\vecN(s) = \frac{1}{\kappa}\frac{d^2\vecx}{ds^2};\ \kappa \equiv \lvert\frac{d^2\vecx}{ds^2}\rvert;$$
  次法向量(binormal vector)
  $$\vecB(s) = \vecT(s)\times \vecN(s).$$

  {\scriptsize 注:垂直于$\vecT(s)$, $\vecN(s)$, $\vecB(s)$ 的平面分别称为法平面,切平面,和密切平面。}
\end{frame}



\begin{frame}
  \frametitle{嵌在三维空间的曲线的挠率(torsion)}
  一般来说当沿着曲线运动时,密切平面会发生“摆动”,即次法向量 $\vecB$ 会发生变化。我们先来证明,$\frac{d\vecB}{ds}$ 一定和 $\vecN$ 同向。

  把归一化条件 $\vecB^2 = 1$ 对 $s$ 求导,可得到 $\frac{d\vecB}{ds} \cdot \vecB = 0$,即 $\frac{d\vecB}{ds}$ 和 $\vecB$ 是垂直的。此外,
  $$\frac{d\vecB}{ds} = \frac{d(\vecT\times \vecN)}{ds} = \frac{d\vecT}{ds}\times \vecN + \vecT \times \frac{d\vecN}{ds}$$
  注意到 $\frac{d\vecT}{ds} = \kappa \vecN$,上式右边的第一项为零,即 $\frac{d\vecB}{ds}$ 和 $\vecT$ 也垂直。
  
  于是不妨设
  $$\frac{d\vecB}{ds} = -\tau \vecN.$$
  我们把 $\tau$ 称为{\blue 挠率}。注意和曲率不同,挠率是可正可负的。它的大小 $|\tau|= \lvert \frac{d\vecB}{ds}\rvert$ 刻画了密切平面的摆动快慢。
\end{frame}

\thinkf{如果一条曲线的挠率处处为零,这说明了曲线的什么几何特性?}


\begin{frame}
  \frametitle{嵌在三维空间的曲线的Frenet方程}
  我们来研究Frenet标架的变化率。首先,根据定义已经知道了:
  $$\frac{d\vecT}{ds} = \kappa \vecN$$
  和
  $$\frac{d\vecB}{ds} = -\tau \vecN.$$  
  剩下的非常简单:
  $$\frac{d\vecN}{ds} = \frac{d\vecB}{ds}\times \vecT + \vecB\times\frac{d\vecT}{ds} =-\tau\vecN\times\vecT +\kappa\vecB\times\vecN = \tau\vecB - \kappa \vecT.$$
  
\end{frame}


\begin{frame}
  \frametitle{把Frenet方程写成矩阵形式}
  \be
  \frac{d}{ds}
  \begin{pmatrix}
    \vecT \\
    \vecN \\
    \vecB
  \end{pmatrix}
  =
  \begin{pmatrix}
    0 & \kappa & 0 \\
    -\kappa & 0 & \tau \\
    0 & -\tau & 0
  \end{pmatrix}
  \begin{pmatrix}
    \vecT \\
    \vecN \\
    \vecB
  \end{pmatrix}
  \ee
\end{frame}


\begin{frame}
  \frametitle{思考题(本题给出了挠率的一种常见计算方法)}
  设 $\vecx(s)$ 是三维空间的一条曲线,$s$是弧长参数。试证明:以三个矢量 $\frac{d\vecx}{ds}$, $\frac{d^2\vecx}{ds^2}$, $\frac{d^3\vecx}{ds^3}$ 为列矢量的行列式的值等于 $\tau \kappa^2$。
\end{frame}


\begin{frame}
  \frametitle{如果曲线不是用弧长参数表示的怎么办}
  一般参数表示的曲线 $\vecx(t)$ 的Frenet标架、曲率、挠率的计算都比较复杂。但都可以用
  $$\frac{ds}{dt} = \lvert\frac{d\vecx}{dt}\rvert$$
  进行转化。下面我们直接给出计算结果:
\end{frame}


\begin{frame}
  \frametitle{一般参数曲线$\vecx(t)$的Frenet计算公式}
  \begin{eqnarray}
    \vecT(t) &=& \frac{\vecx'(t)}{\lvert\vecx'(t)\rvert} \newl
    \vecN(t) &=& \frac{\lvert\vecx'(t)\rvert^2\vecx''(t)-\left(\vecx'(t)\cdot\vecx''(t)\right)\vecx'(t)}{\lvert\vecx'(t)\rvert\,\lvert\vecx'(t)\times\vecx''(t)\rvert } \newl    
    \vecB(t) &=& \frac{\vecx'(t)\times \vecx''(t)}{\lvert\vecx'(t)\times\vecx''(t)\rvert} \newl
    \kappa(t) &=& \frac{\lvert\vecx'(t)\times\vecx''(t)\rvert}{\lvert\vecx'(t)\rvert^3} \newl
    \tau(t) &=& \frac{\vecx'(t)\cdot(\vecx''(t)\times\vecx'''(t))}{\lvert\vecx'(t)\times\vecx''(t)\rvert^2} \nonumber 
  \end{eqnarray}

  请参考 \url{http://zhiqihuang.top/gr/codes/frenet.py}
\end{frame}


\begin{frame}
  \frametitle{曲线的邻域展开}
  由于 $\vecT$, $\vecN$, $\vecB$ 包含了曲线的一次导数,二次导数,三次导数的信息,我们可以在一点附近展开曲线到三阶近似:

  \bea
  \vecx(s+ds) & \approx & \vecx(s) \newl
  && + \left(ds - \frac{\kappa^2}{6}ds^3\right)\vecT(s) \newl
  &&  + \left(\frac{\kappa}{2}ds^2 + \frac{1}{6}\frac{d\kappa}{ds}ds^3\right)\vecN(s) \newl
  && + \frac{\kappa\tau}{6}ds^3\vecB(s). 
  \eea  
\end{frame}


\begin{frame}
  \frametitle{例题:球面上的曲线}
  设 $\vecx(s)$ 是三维空间的球面上的一条曲率和挠率均非零的曲线,$s$ 是弧长参数。证明:
  $$\frac{1}{\kappa^2} +\left[\frac{1}{\tau}\frac{d}{ds}\left(\frac{1}{\kappa}\right)\right]^2$$
  是常数。

\end{frame}


\begin{frame}
  \frametitle{解法1}
  证明:不妨设球半径为单位长度,并取球心为原点。球面上的曲线满足方程:
  \begin{equation}
  \vecx^2 = 1.
  \end{equation}
  两边对 $s$ 求导得到
  \begin{equation}
  \vecx\cdot\vecT = 0.
  \end{equation}
  于是可以假设
  \begin{equation}
  \vecx = \alpha \vecN + \beta \vecB
  \end{equation}
  其中 $\alpha$, $\beta$ 是待定的(依赖于$s$的)参量,满足 $\alpha^2+\beta^2=1$。两边继续对 $s$ 求导并利用Frenet公式,
  \begin{equation}
    \vecT =-\alpha\kappa\vecT + \left(\frac{d\alpha}{ds}-\tau\beta\right)\vecN + \left(\frac{d\beta}{ds} + \alpha\tau\right)\vecB.
  \end{equation}
\end{frame}

\begin{frame}
  \frametitle{解法1}
  于是有
  \begin{eqnarray}
    \alpha &=& -\frac{1}{\kappa};  \\
    \beta &=& -\frac{1}{\tau}\frac{d}{ds}\left(\frac{1}{\kappa}\right).
  \end{eqnarray}
  于是利用 $\alpha^2+\beta^2=1$ 即得证。
\end{frame}


\begin{frame}
  \frametitle{解法2}
  证明:不妨设球半径为单位长度,并取球心为原点。球面上的曲线满足方程:
  \begin{equation}
  \vecx^2 = 1.
  \end{equation}
  把 $\vecx(s+ds)^2 = 1$ 近似展开到 $ds^3$量级,有
  \bea
  1 &=& 1 + (2ds -\frac{\kappa}{3}ds^3)\vecx(s)\cdot\vecT(s) \newl
  && +\left(\kappa ds^2 + \frac{1}{3}\frac{d\kappa}{ds}ds^3\right)\vecx(s) \cdot \vecN(s) \newl
  && + \frac{\kappa\tau}{3}ds^3 \, \vecx(s)\cdot\vecB(s) \newl
  && + ds^2.
  \eea  

\end{frame}

\begin{frame}
  \frametitle{解法2}
  由 $ds$一次项系数为零,得到
  $$\vecx(s)\cdot\vecT(s)=0.$$  
  由 $ds^2$系数为零,得到
  $$\vecx(s)\cdot\vecN(s) = -\frac{1}{\kappa}.$$
  再由 $ds^3$ 系数为零,得到
  $$\vecx(s)\cdot\vecB(s) = \frac{1}{\kappa^2\tau}\frac{d\kappa}{ds} = -\frac{1}{\tau}\frac{d}{ds}\left(\frac{1}{\kappa}\right).$$
  于是得证。
\end{frame}


\begin{frame}
  \frametitle{曲线论基本定理}
  根据我们对曲率和挠率的几何图像理解,下列定理是显然的:

  \skipline
  
  {\blue 三维欧氏空间内两条以弧长 $s$ 为参数的正则曲线。如果它们的曲率处处不为零,且对同一个 $s$,两条曲线的曲率,挠率都分别相等。则两条曲线可以通过一个刚体运动重叠起来。}

\end{frame}


\begin{frame}
  \frametitle{例题}
  证明曲线 $\vecx(t) = (\cosh t, \sinh t, t)$ 和曲线 $\vecy(u) = \left(\frac{e^{-u}}{\sqrt{2}}, \frac{e^u}{\sqrt{2}}, u+1\right)$ 在三维欧式空间内的一个刚体运动下是重合的。
\end{frame}



\begin{frame}
  \frametitle{解答}
  证明: 直接计算得到 $\vecx(t)$ 弧长参数,曲率和挠率为:
  $$ s_x(t) = \sqrt{2} \sinh t, \kappa_x(t) = \tau_x(t) = \frac{1}{2\cosh^2(t)} $$
  $\vecy(u)$ 的弧长参数,曲率和挠率为
  $$ s_y(u)  = \sqrt{2} \sinh u, \kappa_y(u) = \tau_y(u) = \frac{1}{2\cosh^2(u)}.$$
  故同样的 $t$ 和 $u$ 对应相同的弧长参数,曲率,挠率。证毕。
\end{frame}


\begin{frame}
  \frametitle{二维欧式空间中的曲线}
  嵌入在二维欧式空间里的曲线可以看成嵌在三维欧式空间中的无挠(挠率为零的)曲线。平面曲线有一些独特的性质,我们来专门讨论一下。

  设曲线 $\vecr(s) = (x(s), y(s))$ 以弧长 $s$ 为参数。则切向量为
  $$ \vecT(s) = \left(x'(s), y'(s)\right),$$
  逆时针方向旋转$\frac{\pi}{2}$,得到{\blue 相对法向量}为
  $$\vecN_r(s) = \left(-y'(s), x'(s)\right).$$
  注意三维情况定义 $\vecN(s)$ 时我们是按“往那边弯曲就指向哪边”的原则来定义的。所以 $\vecN_r$ 和 $\vecN$ 有可能相同,也可能差一个负号。
  
\end{frame}



\begin{frame}
  \frametitle{相对曲率}
  由相对法向量可以定义``{\blue 相对曲率}'':
  $$ \kappa_r \equiv \vecN_r \cdot \frac{d^2\vecx}{ds^2} $$
  显然它可能和三维情况定义的 $\kappa$ 相同,也可能差一个负号。

  容易看出来,相对曲率可以写成行列式的形式
  \be
  \kappa_r = \left\vert
  \begin{array}{ll}
    x'(s) & y'(s) \\
    x''(s) & y''(s)     
  \end{array}
  \right\vert
  \ee
  它可正可负,而之前定义的 $\kappa = |\kappa_r|$ 一定非负.
\end{frame}

\begin{frame}
  \frametitle{相对曲率}
  对一般的参数形式表示的平面曲线 $\left(x(t), y(t)\right)$,相对曲率可以写成
  \be
  \kappa_r = \frac{1}{\left[x'(t)^2+ y'(t)^2\right]^{3/2}}\left\vert
  \begin{array}{ll}
    x'(t) & y'(t) \\
    x''(t) & y''(t)     
  \end{array}
  \right\vert
  \ee
\end{frame}


\begin{frame}
  \frametitle{平面曲线的 Frenet 方程}
  平面曲线的 Frenet 方程也变得非常简单:

  \be
  \frac{d}{ds}
  \begin{pmatrix}
    \vecT \\
    \vecN_r
  \end{pmatrix}
  =
  \begin{pmatrix}
    0 & \kappa_r \\
    -\kappa_r & 0 
  \end{pmatrix}
  \begin{pmatrix}
    \vecT \\
    \vecN_r
  \end{pmatrix}
  \ee
\end{frame}


\ech
\end{document}
