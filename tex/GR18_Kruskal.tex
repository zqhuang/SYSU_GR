\documentclass[CJK,13pt]{beamer}
\input{macros.tex}
\def\courseurl{http://zhiqihuang.top/gr}

\def\tpage#1#2{
\title{GR \S{#1}  #2}
  \author{Zhiqi Huang}

\begin{frame}
\begin{center}
{\bf \Huge G}eneral {\bf \Huge R}elativity

{\vskip 0.1in}



{\Large \S #1 #2}

{\vskip 0.2in}

{Lecturer: 黄志琦}

\vskip 0.2in

\courseurl

\end{center}
\end{frame}
}


  \date{}
  \begin{document}
  \bch
  \tpage{18}{Kruskal Coordinate and White Hole}

  \begin{frame}
    胎教小练习:证明对 $x\ge -1$ 存在唯一的 $W\ge 0$ 满足 $(W-1)e^W = x$。

    \addfig{2.8}{Wofx.png}
    
    \url{http://zhiqihuang.top/gr/codes/solvew.py}
    
  \end{frame}


  \begin{frame}
    
    Schwarzschild时空可以用很多等价形式表述出来:
    
    \bitem
    \item{turtoise 坐标}
    \item{advanced Eddington-Finkelstein坐标}

    \item{retarded Eddington-Finkelstein坐标}

    \item{Painlev\'{e}坐标}
      
    \item{Kruskal-Szekeres坐标,简称Kruskal坐标(Szekeres:???)}
    \item{Kerr-Schild 坐标}
    \item{\ldots}
      \eitem

  \end{frame}
  

  \begin{frame}
    \frametitle{我们选择Kruskal——因为\sout{拼写简单}好用}

    Kruskal 坐标由 $\tau,\mu,\theta,\phi$ 描述,度规是
    {\blue \small $$ ds^2 = \frac{4e^{-\mathrm{W}\left(\frac{\mu^2-\tau^2}{4G^2M^2}\right)}}{\mathrm{W}\left(\frac{\mu^2-\tau^2}{4G^2M^2}\right)} \left(d\tau^2-d\mu^2\right) - \left[2GM\, \mathrm{W}\left(\frac{\mu^2-\tau^2}{4G^2M^2}\right)\right]^2\left(d\theta^2 + \sin^2\theta d\phi^2\right)$$}
    这里的$\mathrm{W}$ 是我们前面讨论过的 $(x-1)e^x$ 的反函数;另外,我们约定,{\blue 时间的因果方向取为 $+\tau$ 的方向。}

    为了使 $\mathrm{W}$ 的宗量不小于 $-1$,我们限定坐标范围
    $$ \tau^2-\mu^2 \le 4G^2M^2$$

    这在 $\mu$-$\tau$ 平面上给出的范围夹在两条双曲线之内。
    
  \end{frame}

  \begin{frame}
    \frametitle{从Kruskal坐标到Schwarzschild坐标}
    Kruskal坐标没有度规奇异性,因此可以放心拿来做解析或者数值计算。
    
    因为 Schwarzschild 坐标对远距离 ($r\gg GM$) 观测者而言非常适用,所以我们希望最后能把 Kruskal 坐标转化为 Schwarzschild 坐标。

    两个坐标的 $\theta,\phi$ 均相同,而 {\blue
    \be
    r=2GM\, \mathrm{W}\left(\frac{\mu^2-\tau^2}{4G^2M^2}\right)
    \ee}
    特别地,当 $\tau=\pm \mu$, 有 $r=2GM$。牛顿极限 $r\gg 2GM$ 则等价于 $\mu^2-\tau^2 \gg 4G^2M^2$
  \end{frame}

  \begin{frame}
    \frametitle{从Kruskal坐标到Schwarzschild坐标(续)}
    有了 $r$ 之后,时间坐标 $t$ 要分类讨论:

    {\blue
    如果 $r\ge 2GM$,则
    $$\tanh \frac{t}{4GM} =  \frac{\tau}{\mu}$$
    如果 $r\le 2GM$,则
    $$\tanh \frac{t}{4GM} =  \frac{\mu}{\tau}$$    
    }
    
    利用W函数的定义,有{\blue
    \be
    \left(\frac{r}{2GM}-1\right)e^{\frac{r}{2GM}} = \frac{\mu^2-\tau^2}{4G^2M^2}
    \ee
    }
    利用该式和上面 $t$ 的表达式,我们也可以从 $(r, t)$ 映射到 $(\mu, \tau)$ (但会有一些符号不确定性)。
    
  \end{frame}  

  \begin{frame}
    \addfig{3.4}{Kruskal.jpg}
  \end{frame}

  \thinkf{如果白洞和Kruskal坐标图里的“另一个宇宙”确实存在,我们能和“另一个宇宙”进行通话吗?}

  \begin{frame}
    \frametitle{Kruskal坐标系里的光子的径向运动}
    光子走的是零测地线,所以径向运动方程非常简单:
    $$d\mu = \pm d\tau $$
    且这里的 $\pm$ 符号在运动过程中保持不变。

    \skiplines

    也就是说,在 $\tau$-$\mu$ 平面上光子的径向运动世界线都是 $45^\circ$ 直线。
  \end{frame}
    

  \begin{frame}
    \frametitle{Kruskal坐标系里非零质量粒子的径向运动}    
    对质量非零粒子,直接给出我推导的结果:
    \bea
    \mu\frac{d\tau}{ds}&=&\tau\frac{d\mu}{ds} + \varepsilon GM \mathrm{W}e^{\mathrm{W}}  \newl
    \frac{d\tau}{ds}&=& \sqrt{\left(\frac{d\mu}{ds}\right)^2 + \frac{\mathrm{W}e^{\mathrm{W}}}{4}}         
    \eea
    这里的 $\varepsilon$ 是单位质量测试粒子的守恒能量。$\mathrm{W}$ 是 $\mathrm{W}\left(\frac{\mu^2-\tau^2}{4G^2M^2}\right)$ 的简写。

    \skipline
    
    {\scriptsize 为了说服你我的推导是靠谱的,我特地写了一段代码来数值计算沿径向自由下落粒子的 $r(s)$ 函数,并和上一讲我们获得的解析解做比较。

      \url{http://zhiqihuang.top/gr/codes/KruskalFreeFall.py}
    }
  \end{frame}


  \begin{frame}
    \frametitle{忘了带秋裤……}

    \addfig{1}{qiuku.jpg}
    
    上一讲中我的黑洞旅行故事中加一个情节:空间站上的人发现我忘了带秋裤,就想发光信号告诉我。为了让我能收到信号,发信号时间(按空间站上的人的计时)不得晚于我出发后多久?
    
  \end{frame}
  \ech
\end{document}




  
