\documentclass[CJK,13pt]{beamer}
\input{macros.tex}
\def\courseurl{http://zhiqihuang.top/gr}

\def\tpage#1#2{
\title{GR \S{#1}  #2}
  \author{Zhiqi Huang}

\begin{frame}
\begin{center}
{\bf \Huge G}eneral {\bf \Huge R}elativity

{\vskip 0.1in}



{\Large \S #1 #2}

{\vskip 0.2in}

{Lecturer: 黄志琦}

\vskip 0.2in

\courseurl

\end{center}
\end{frame}
}


  \date{}
  \begin{document}
  \bch
  \tpage{14}{Spherical Symmetric Metric}

  \thinkf{证明爱因斯坦方程
    $$ R_{\mu\nu}-\frac{1}{2}g_{\mu\nu}R = 8\pi GT_{\mu\nu}$$
    可以等价地写成
    $$R_{\mu\nu} = 8\pi GT_{\mu\nu} - 4\pi GT g_{\mu\nu}$$
    这里的 $T\equiv T^\rho_{\ \ \rho}$.
  }

  \begin{frame}
    本讲的任务是研究 \sout{真空中的球形火鸡} 球对称的度规

    \addfig{1}{sphericalchicken.jpg}
  \end{frame}
  
  \begin{frame}
    \frametitle{球对称的度规是指}

    三个空间转动映射的生成元都是Killing vector……

    \addfig{1.3}{shuorenhua.jpg}
    
  \end{frame}

  \begin{frame}
  \frametitle{球对称的度规是指}

  \sout{三个空间转动映射的生成元都是Killing vector……}

  \skipline
  
    可以取坐标系 $(t, r, \theta,\phi)$,使得
   {\blue $$ ds^2 = e^{2\Phi(r,t)}dt^2 - e^{-2\Psi(r, t)}dr^2 - r^2\left(d\theta^2 + \sin^2\theta d\phi^2\right)$$}
    
    
  \end{frame}

  \begin{frame}
    \frametitle{Ricci张量}
    用 sympy 代码 \url{http://zhiqihuang.top/gr/codes/spherical.py} 以及 latex 编译器导出
    {\small

\begin{eqnarray}
R_{00} &=&  e^{2 \Phi + 2 \Psi} \left[\left(\Phi_{,r}\right)^{2} +  \Phi_{,r} \Psi_{, r} + \Phi_{,r,r}+\frac{2}{r}  \Phi_{,r}\right] - \Phi_{,t} \Psi_{,t} - \left(\Psi_{,t}\right)^{2} + \Phi_{,t,t}  \nonumber \\
R_{10} &=& - \frac{2}{r} \Psi_{,t} \nonumber \\
R_{11} &=&  e^{- 2 \Phi- 2 \Psi} \left[\Phi_{,t} \Psi_{,t} +  \left(\Psi_{,t}\right)^{2} -  \Phi_{,t,t} \right]- \left(\Phi_{,r}\right)^{2} - \Phi_{,r} \Psi_{, r} - \Phi_{,r,r} - \frac{2}{r} \Psi_{, r} \nonumber \\
R_{22} &=& 1 - r e^{2 \Psi} \Phi_{,r} - r e^{2 \Psi} \Psi_{, r} - e^{2 \Psi}  \nonumber \\
R_{33} &=& R_{22} \sin^2\theta \nonumber 
\end{eqnarray}
    }
    这些方程是我们下面讨论的出发点。
  \end{frame}

  \secpage{真空中的球形……解}{$$ ds^2 = \left(1-\frac{2GM}{r}\right)d\tau^2 - \frac{1}{1-\frac{2GM}{r}} dr^2 - r^2\left(d\theta^2 +\sin^2\theta d\phi^2\right)$$}
  
  \begin{frame}
    真空中的  Einstein 方程为 $R_{\mu\nu}=0$。由 $R_{10}=0$ 立刻得到: $\Psi_{,t}=0$,即 $\Psi$ 只依赖于$r$。

    再根据 $R_{00}= R_{11}= R_{22}=0$ ($R_{33}=R_{22}\sin^2\theta$ 不需要另外考虑) 得到:
    \begin{eqnarray}
r\left[\left(\Phi_{,r}\right)^{2} +  \Phi_{,r,r}+ \Phi_{,r} \Psi_{, r}  \right] +  2\Phi_{,r}&=& 0 \label{eq:00}\\
-r \left[\left(\Phi_{,r}\right)^{2} + \Phi_{,r} \Psi_{, r} + \Phi_{,r,r}\right] - 2 \Psi_{, r} &=&0 \label{eq:11} \\
e^{-2 \Psi}- r \left(\Phi_{,r} + \Psi_{, r}\right)  - 1  &=& 0 \label{eq:22}
    \end{eqnarray}
  \end{frame}

  \begin{frame}
方程 \eqref{eq:00} + 方程 \eqref{eq:11},得到 $(\Phi - \Psi)_{,r}=0$,积分即得到
\begin{equation}
  \Phi - \Psi = 2\lambda(t), \nonumber
\end{equation}
这里的 $\lambda(t)$ 为任意函数。我们可以令 $d\tau =e^{\lambda(t)}dt$, 把度规写成
{\blue $$ ds^2 = e^{2\Psi(r)} d\tau^2 - e^{-2\Psi(r)} dr^2 - r^2(d\theta^2+\sin^2\theta d\phi^2)$$}

\skipline

显然,剩下的任务是搞定 $\Psi(r)$。
  \end{frame}

  \begin{frame}
    方程 \eqref{eq:22} 里的 $\Phi_{,r} = \Psi_{,r}$,所以
    $$\frac{2\Psi_{,r}}{e^{-2\Psi}-1} = \frac{1}{r}$$
    令 $e^{2\Psi}= 1- f$,方程变为
    $$\frac{f'}{f} = -\frac{1}{r}$$
    故
    $$  f = \frac{2GM}{r}$$
    这里的 $G$ 为引力常数, $M$为待定常数。我们马上会看到用 $2GM$这样的符号来表示积分常数的用意。
  \end{frame}

  \begin{frame}
    \frametitle{史瓦西度规}
    
    于是,球对称的真空度规(假设取了合适的时间坐标)一定可以写成
    \tbox{$$ ds^2 = \left(1-\frac{2GM}{r}\right)d\tau^2 - \frac{1}{1-\frac{2GM}{r}} dr^2 - r^2\left(d\theta^2 +\sin^2\theta d\phi^2\right)$$}
    这个度规叫史瓦西 (Schwarzschild) 度规。


    \bmini{0.3}
    \lfig{1}{Schwarzschild.jpg}
    \emini
    \bmini{0.65}
          {\small Schwarzschild 是个神奇的牛人,有兴趣的童鞋可以去搜索了解下}
    \emini

  \end{frame}


  \begin{frame}
    \frametitle{伯克霍夫定理}
    
    在推导过程中我们并没有假设这个度规是静态的(和时间 $\tau$ 无关),所以实际上我们还顺手证明了伯克霍夫(Birkhoff) 定理(史瓦西解不需要静态假设)。


    \bmini{0.3}
    \lfig{1}{George_David_Birkhoff.jpg}
    \emini
    \bmini{0.65}
          大家好,我叫Birkhoff,我对物理的贡献是:证明了在内心zuo基本上是没有用的。
    \emini

  \end{frame}


  \begin{frame}
    \frametitle{下面我们来说明——为什么 $M$ 是\sout{火鸡}引力源的质量}
    在度规是静态、在空间变化缓慢、且非常接近 Minkowski 度规的情况下,非相对论运动的 ,静质量非零的粒子的运动方程近似为

    $$ \frac{d^2x^i}{dt^2}\approx -\Gamma^i_{\ 00}\approx \Gamma_{i00} \approx -\frac{1}{2}g_{00,i}$$

    也就是 $\frac{g_{00}-1}{2}$ (减去 $1$ 是对应 Minkowski 情况的引力势消失) 起到于引力势的作用。

    \skipline

    对照前面的史瓦西度规,就明白为啥 $M$ 对应引力源的质量了吧。
    
  \end{frame}

  \secpage{静态球对称的恒星内部}{$$\frac{dp}{dr} = -\frac{(\rho+p)(4\pi G pr^3+Gm)}{r(r-2Gm)} $$}
  
  \begin{frame}
    对静态球对称的恒星,度规可以设为
    $$ ds^2 = e^{2\Phi(r)}dt^2 - e^{-2\Psi(r)}dr^2 - r^2\left(d\theta^2 + \sin^2\theta d\phi^2\right)$$

    来来跑下代码 \url{http://zhiqihuang.top/gr/codes/sphericalstatic.py} 搞定:
    \begin{eqnarray}
G^0_{\ 0} &=& \frac{1}{r^{2}} \left(- 2 r e^{2 \Psi} \Psi_{, r} - e^{2 \Psi} + 1\right) \nonumber \\
G^1_{\ 1} &=& \frac{1}{r^{2}} \left(- 2 r e^{2 \Psi} \Phi_{,r} - e^{2 \Psi} + 1\right) \nonumber \\
G^2_{\ 2} &=& - \frac{1}{r} \left(r \left(\Phi_{,r}\right)^{2} + r \Phi_{,r} \Psi_{, r} + r \Phi_{,r,r} + \Phi_{,r} + \Psi_{, r}\right) e^{2 \Psi} \nonumber \\
G^3_{\ 3} &=& G^2_{\ 2} \nonumber
    \end{eqnarray}
  \end{frame}


  \begin{frame}
    把恒星近似当作静态球对称的理想流体,其每个宏观流体元的四维速度都是 $u^\mu = (\frac{1}{\sqrt{g_{00}}},0,0,0)$。根据
    $$T^\mu_{\ \nu} = (\rho+p)u^\mu u_\nu - pg^\mu_{\ \ \nu}$$
    恒星内部的能量动量张量为:
    \be
      T^{\mu}_{\ \ \nu} = \diag\left(\rho, -p, -p, -p\right)
    \ee
    这里的 $\rho, p$ 分别是能量密度和压强,根据球对称性它们只依赖于 $r$。
  \end{frame}

  \begin{frame}
    具体给定的物质会有个“状态方程”:
    \begin{equation}
      p =  p(\rho) \label{eq:eos}
    \end{equation}
    即 $p$ 可由 $\rho$ 确定(例如光子气体的 $p=\rho/3$),所以在假设我们知道恒星的物质组成的情况下, $\rho, p$ 仅仅看成一个未知函数。此外,压强在恒星边界处(设为 $r=a$)消失:
      \begin{equation}
       \left. p\right\vert_{r=a} =0 \label{eq:pbound}
      \end{equation}
      在边界处的 $\Phi$ 和 $\Psi$ 和外部的史瓦西解衔接
      \begin{equation}
       \left. e^\Phi\right\vert_{r=a} = \left. e^\Psi\right\vert_{r=a} = 1-\frac{2GM}{r}  \label{eq:pbound}
      \end{equation}
      这里的等效质量 $M$,也就是恒星外部的牛顿力学观测者得到的恒星质量,具体和 $\rho$ 的关系如何,暂时还不明确。
      
  \end{frame}


  \begin{frame}
    Einstein方程给出
    {\small
    \begin{eqnarray}
    \frac{1}{r^{2}} \left(- 2 r e^{2 \Psi} \Psi_{, r} - e^{2 \Psi} + 1\right) &=& 8\pi G\rho \label{eq:G00} \\    
    \frac{1}{r^{2}} \left(- 2 r e^{2 \Psi} \Phi_{,r} - e^{2 \Psi} + 1\right) &=& -8\pi Gp \label{eq:G11} \\
    - \frac{1}{r} \left(r \left(\Phi_{,r}\right)^{2} + r \Phi_{,r} \Psi_{, r} + r \Phi_{,r,r} + \Phi_{,r} + \Psi_{, r}\right) e^{2 \Psi} &=& -8\pi Gp    \label{eq:G22}
    \end{eqnarray}
    }
    由 Bianchi 恒等式, Einstein 方程已经包含了能量动量张量的守恒方程,但是通常这需要一波操作才能看出来。

  \end{frame}
  
  \begin{frame}
    手滑党表示扛不住操作,不如直接写 $T^\mu_{\ \ \nu;\mu} = 0$。    

    \skipline

    老规矩,跑下代码 \url{http://zhiqihuang.top/gr/codes/sphericalT.py}
    
    发现只有 $T^\mu_{\ r;\mu} =0$ 是非平凡的,给出
    \begin{equation}
      p_{,r} = -(p + \rho) \Phi_{,r} \label{eq:Tcons}
    \end{equation}

  \end{frame}

  \begin{frame}
    方程 \eqref{eq:G00} 可以写成
    \begin{equation}
      \frac{d}{dr} \left[\frac{r}{2G}\left(1-e^{2\Psi}\right)\right] = 4\pi r^2\rho \nonumber
    \end{equation}
    也就是我们可以得到
    \begin{equation}
      e^{2\Psi} = 1-\frac{2G\,m(r)}{r} \label{eq:Psi}
    \end{equation}
    这里的 $m(r)$ 是``累计等效质量'',对任意 $0\le q\le a$,有
    \begin{equation}
      m(q) = \int_0^q 4\pi \rho r^2 dr \label{eq:mdef}
    \end{equation}
        {\scriptsize 注意: 它出乎意料地不等于累计能量 $E(q) = \int_0^q 4\pi \rho r^2e^{-\Psi}dr$.}
        
    由此即确定了
    \begin{equation}
      M = m(a) = \int_0^a 4\pi \rho r^2 dr \label{eq:M}
    \end{equation}
  \end{frame}
  
  \begin{frame}
    方程 \eqref{eq:G11} - 方程 \eqref{eq:G00},得到
    $$ \frac{2}{r}  e^{2 \Psi} \left(\Phi_{,r}-\Psi_{, r}\right)= 8\pi G(\rho+p).$$
    即
    \begin{equation}
      \Phi_{,r} = \Psi_{,r} + 4\pi G(\rho + p) r e^{-2\Psi} = \frac{4\pi Gp r^3+G\,m(r)}{r\left[r-2G\,m(r)\right]}\label{eq:Phir}
    \end{equation}
    这个结果代到 Eq.~\eqref{eq:Tcons},即得到了大名鼎鼎的奥本海默(Oppenheimer)方程:
    \tbox{$$\frac{dp}{dr} = -\frac{\left(\rho+p\right)\left[4\pi Gp r^3+G\,m(r)\right]}{r\left[r-2G\,m(r)\right]}$$}
  \end{frame}

  \begin{frame}
    总结求解半径为 $a$ 的恒星内部物质分布和引力场的方法:把状态方程($\rho$ 和 $p$ 的互相依赖关系)代入奥本海默方程里
    \tbox{$$\frac{dp}{dr} = -\frac{\left(\rho+p\right)\left[4\pi Gp r^3+G\,m(r)\right]}{r\left[r-2G\,m(r)\right]}$$}
    这里的 {\blue $m(q) \equiv \int_0^q 4\pi \rho(r) r^2 dr$}。利用上式和边界条件 {\blue $p(a)=0$} 求解出 $p(r), \rho(r)$。

    \skipline

    知道 $\rho(r)$ 之后,方程 \eqref{eq:Psi} 立刻就给出了{\blue $e^{2\Psi(r)}= 1-\frac{2G\,m(r)}{r} $}。

    \skipline

    然后对方程 \eqref{eq:Phir},也就是{\blue
    \begin{equation}
      \frac{d\Phi}{dr} =  \frac{4\pi Gp r^3+G\,m(r)}{r\left[r-2G\,m(r)\right]}\nonumber
    \end{equation}}
    进行积分,利用边界上{\blue $\Phi(a)=\Psi(a)$} 可以完全确定 $\Phi(r)$。

    最后,如果从外面观测,恒星的等效质量 $M=m(a)$.
  \end{frame}

  \begin{frame}

    这一讲我们体验了人机结合的强大之处,连续轻松爆掉史瓦西,伯克霍夫,奥本海默三位巨佬!

    \addfig{2}{renji.jpg}
    
    是不是有点小激动,准备去研究下中子星的状态方程呢?
    
  \end{frame}  
\ech
\end{document}




  
