\documentclass[CJK,13pt]{beamer}
\input{macros.tex}
\def\courseurl{http://zhiqihuang.top/gr}

\def\tpage#1#2{
\title{GR \S{#1}  #2}
  \author{Zhiqi Huang}

\begin{frame}
\begin{center}
{\bf \Huge G}eneral {\bf \Huge R}elativity

{\vskip 0.1in}



{\Large \S #1 #2}

{\vskip 0.2in}

{Lecturer: 黄志琦}

\vskip 0.2in

\courseurl

\end{center}
\end{frame}
}


  \date{}
  \begin{document}
  \bch
\tpage{0}{Introduction}


\begin{frame}
\frametitle{教材和参考书目}

教材:
\tbox{《广义相对论引论》第2版,俞允强著,北京大学出版社; ISBN:9787301033173}

\skiplines

\sout{参考书目} 年轻人,你渴望力量吗?
\bitem
\item{《引力论和宇宙论 - 广义相对论的原理和应用》,S.温伯格 著,邹振隆、张历宁等 译,高等教育出版社,ISBN: 9787040487183}
\item{《General Theory of Relativity》, P.A.M.~Dirac 世界图书出版社 影印版, ISBN: 9780691011462}
\item{《微分几何初步》第1版,陈维桓,北京大学出版社;ISBN:9787301012291}
\eitem
      
\end{frame}




\begin{frame}
\frametitle{评分标准}

\tbox{作业 $30\%$ + 课堂表现 $20\%$ + 期末闭卷考试 $50\%$}

\skipline

课后作业: http://zhiqihuang.top/gr

\end{frame}




\begin{frame}
  \frametitle{课程难度}

  \lfig{1}{sheep.jpg}  \lfig{1}{sheep.jpg}  \lfig{1}{sheep.jpg}

  一只羊,两只羊,三只羊……


  \skiplines
  
  (全程幼儿园难度,大佬请自觉退课)
  
\end{frame}


\begin{frame}
  \frametitle{课程结构}
  \tbox{{\bf 常识} 狭义相对论,曲线和曲面论}
  \tbox{{\bf 两点之间啥线最短} 度规和测地线}
  \tbox{{\bf 数数上下标} 张量分析基础}
  \tbox{{\bf 一个方程} 爱因斯坦方程}
  \tbox{{\bf 不会的都扔掉} 牛顿近似和后牛顿近似}
  \tbox{{\bf 吹牛} 引力波、黑洞和宇宙学}  
\end{frame}


\ech
\end{document}
